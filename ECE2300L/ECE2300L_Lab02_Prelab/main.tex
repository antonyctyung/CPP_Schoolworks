\documentclass{article}
\usepackage[utf8]{inputenc}
\usepackage{siunitx}
\usepackage[american,siunitx]{circuitikz}
\usepackage{amsmath}
\usepackage{svg}
\usepackage{booktabs}
\usepackage{float}
\usepackage{xparse, xfp}
\renewcommand{\thesubsection}{\thesection.\alph{subsection}}
\newcommand{\equal}{=}

\title{ECE2300L_Prelab2}
\author{Choi Tim Antony Yung}
\begin{document}

\begin{center}
    \begin{circuitikz}
        \draw
        (0,0) node[xor port](xor1){86}
        ++(0,-2)node[and port](and){08}
        (and.in 1) ++(-0.5,0) coordinate (crossing) ++(-0.5,0) coordinate(b4xing)
        (xor1.in 1) -- ++(-1,0) node[circ,label=A]{} -- (b4xing) to[xing] (and.in 1)
        (xor1.in 2) -- ++(-0.5,0) coordinate (temp) -- (temp|-and.in 2) node[circ,label=270:B]{} -- (and.in 2)
        (temp|-and.in 2) -- ++(-1,0) to [nos] ++(0,1) node[vcc] {5V}
        (temp|-and.in 2) -- ++(-1,0) -- ++(0,-0.5) to[Do] ++(0,-1) to[R=\SI{330}{\ohm}] ++(0,-1.5) node[ground]{}
        (xor1.in 1) -- ++(-3,0) to [nos] ++(0,1) node[vcc] {5V}
        (xor1.in 1) -- ++(-3,0) -- ++(0,-0.5) to[Do] ++(0,-1) to[R=\SI{330}{\ohm}] ++(0,-1.5) node[ground]{}
        (2,-1) node[xor port](xor2){86}
        (xor1.out) -- (xor1.out-|xor2.in 1) node[circ,label=right:$A\oplus B$]{} -- (xor2.in 1)
        (and.out) -- (and.out-|xor2.in 2)node[circ,label=right:$AB$]{} -- (xor2.in 2)
        (xor2.out) -- ++(0.5,0) node[circ,label=right:$AB\oplus (A\oplus B)$]{} -- ++(0,-0.5) to[Do] ++(0,-1) to[R=\SI{330}{\ohm}] ++(0,-1.5) node[ground]{}
        ;
    \end{circuitikz}
\end{center}

\begin{displaymath}
    \begin{array}{|c c|c|c|c|c|}
        % |c c|c| means that there are three columns in the table and
        % a vertical bar ’|’ will be printed on the left and right borders,
        % and between the second and the third columns.
        % The letter ’c’ means the value will be centered within the column,
        % letter ’l’, left-aligned, and ’r’, right-aligned.
        A & B & A\oplus B & AB & AB\oplus(A\oplus B) & A+B\\ 
        % Use & to separate the columns
        \hline  
        % Put a horizontal line between the table header and the rest.
        0 & 0 & 0 & 0 & 0 & 0\\
        0 & 1 & 1 & 0 & 1 & 1\\
        1 & 0 & 1 & 0 & 1 & 1\\
        1 & 1 & 0 & 1 & 1 & 1\\
    \end{array}
\end{displaymath}
\,\\
As seen in the truth table above, $F=AB\oplus(A\oplus B)$ resembles $A+B$, the or gate.

\end{document}
