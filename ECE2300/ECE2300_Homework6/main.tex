\documentclass{article}
\usepackage{multirow}
\usepackage{adjustbox}
\usepackage[table]{xcolor}
\usepackage[utf8]{inputenc}
\usepackage{siunitx}
\usepackage[american,siunitx]{circuitikz}
\usepackage{amsmath}
\usepackage{svg}
\usepackage{booktabs}
\usepackage{float}
\usepackage{xparse, xfp}
\usepackage{tikz}
\usetikzlibrary{calc}
\makeatletter
\newcommand\currcoor{\the\tikz@lastxsaved,\the\tikz@lastysaved}
\makeatother
%\usepackage[landscape]{geometry}
\renewcommand{\thesubsection}{\thesection.\alph{subsection}}
\newcommand{\equal}{=}
\newcommand{\getamag}[2]{
    \tikzset{magcomp/.style={muxdemux,muxdemux def={Lh= 15,Rh= 15,NL=15,NR=9, w=6},no input leads}}
    \draw
        #2 node[magcomp](mag#1){}
        
        (mag#1.blpin  2) node[right]{\large$A_3$}  -- (mag#1.lpin  2) coordinate (#1A3)
        (mag#1.blpin  3) node[right]{\large$A_2$}  -- (mag#1.lpin  3) coordinate (#1A2)
        (mag#1.blpin  4) node[right]{\large$A_1$}  -- (mag#1.lpin  4) coordinate (#1A1)
        (mag#1.blpin  5) node[right]{\large$A_0$}  -- (mag#1.lpin  5) coordinate (#1A0)
        (mag#1.blpin  7) node[right]{\large$B_3$}  -- (mag#1.lpin  7) coordinate (#1B3)
        (mag#1.blpin  8) node[right]{\large$B_2$}  -- (mag#1.lpin  8) coordinate (#1B2)
        (mag#1.blpin  9) node[right]{\large$B_1$}  -- (mag#1.lpin  9) coordinate (#1B1)
        (mag#1.blpin 10) node[right]{\large$B_0$}  -- (mag#1.lpin 10) coordinate (#1B0)
        (mag#1.blpin 12) node[right]{\large$a>b$} -- (mag#1.lpin 12) coordinate (#1alb)
        (mag#1.blpin 13) node[right]{\large$a=b$} -- (mag#1.lpin 13) coordinate (#1aeb)
        (mag#1.blpin 14) node[right]{\large$a<b$} -- (mag#1.lpin 14) coordinate (#1asb)

        (mag#1.brpin 4) node[left]{\Large{$A>B$}} -- (mag#1.rpin 4) coordinate (#1ALB)
        (mag#1.brpin 5) node[left]{\Large{$A=B$}} -- (mag#1.rpin 5) coordinate (#1AEB)
        (mag#1.brpin 6) node[left]{\Large{$A<B$}} -- (mag#1.rpin 6) coordinate (#1ASB)
    ;
}
\newcommand{\getabigfatmag}[2]{
    \tikzset{magcomp/.style={muxdemux,muxdemux def={Lh= 15,Rh= 15,NL=15,NR=9, w=6},no input leads}}
    \draw
        #2 node[magcomp](mag#1){}
        
        (mag#1.blpin  3) node[right]{\large$A_0-A_3$} coordinate (#1bA) ++(-3,0) coordinate(#1A)
        (mag#1.blpin  8) node[right]{\large$B_0-B_3$} coordinate (#1bB) ++(-3,0) coordinate(#1B)

        (mag#1.blpin 12) node[right]{\large$a>b$} -- (mag#1.lpin 12) coordinate (#1alb)
        (mag#1.blpin 13) node[right]{\large$a=b$} -- (mag#1.lpin 13) coordinate (#1aeb)
        (mag#1.blpin 14) node[right]{\large$a<b$} -- (mag#1.lpin 14) coordinate (#1asb)

        (mag#1.brpin 4) node[left]{\large{$A>B$}} -- (mag#1.rpin 4) coordinate (#1ALB)
        (mag#1.brpin 5) node[left]{\large{$A=B$}} -- (mag#1.rpin 5) coordinate (#1AEB)
        (mag#1.brpin 6) node[left]{\large{$A<B$}} -- (mag#1.rpin 6) coordinate (#1ASB)

        (#1bA) -- ++(-1.2,1.2) -- ++(0,-0.8) -- ++(-1.8,0)
        (#1bA) -- ++(-1.2,-1.2) -- ++(0,+0.8) -- ++(-1.8,0)

        (#1bB) -- ++(-1.2,1.2) -- ++(0,-0.8) -- ++(-1.8,0)
        (#1bB) -- ++(-1.2,-1.2) -- ++(0,+0.8) -- ++(-1.8,0)

    ;
}
\newcommand{\greyrule}{\arrayrulecolor{black!30}\midrule\arrayrulecolor{black}}
\newcolumntype{:}{@{\hskip\tabcolsep\color{black!30}\vrule\hskip\tabcolsep}}
\ExplSyntaxOn
\NewExpandableDocumentCommand \groupify { O{\,\allowbreak} m m }
  { \jakob_groupify:nnn {#1} {#2} {#3} }
\cs_new:Npn \jakob_groupify:nnn #1 #2 #3
  { \__jakob_groupify_loop:nnw { 1 } {#2} #3 \q_recursion_tail {#1} \q_recursion_stop }
\cs_new:Npn \__jakob_groupify_loop:nnw #1 #2 #3
  {
    \quark_if_recursion_tail_stop:n {#3}
    \exp_not:n {#3}
    \int_compare:nNnTF {#1} = {#2}
      { \__jakob_groupify_sep:n }
      { \exp_args:Nf \__jakob_groupify_loop:nnw { \int_eval:n { #1+1 } } }
          {#2}
  }
\cs_new:Npn \__jakob_groupify_sep:n #1 #2 \q_recursion_tail #3
  {
    \tl_if_empty:nF {#2} { \exp_not:n {#3} }
    \__jakob_groupify_loop:nnw { 1 } {#1}
    #2 \q_recursion_tail {#3}
  }
\ExplSyntaxOff

\title{ECE 2300\\Digital Logic Design\\\,\\Homework 6}
\author{Choi Tim Antony Yung}
\begin{document}
\maketitle

\thispagestyle{empty}
\setcounter{page}{0}

\newpage

% Q1
\section{Using a magnitude comparator, generate a logic one “Round” signal (to rounda number up by one) if the base 12 input, $N={N_3,N_2,N_1,N_0}$, is six or higher. Let N connect to the comparator's A input.}
$$Round=N>5=(N_3N_2N_1N_0)_2>(0101)_2$$
\,\\

\begin{center}
\begin{circuitikz}
    \getamag{}{(0,0)}
    \draw
        (A3) node[left]{$N_3$}
        (A2) node[left]{$N_2$}
        (A1) node[left]{$N_1$}
        (A0) node[left]{$N_0$}

        (B2) node[above]{$1$} to[crossing] ++(-1,0) node[vcc](vcc){+V}
        (asb) node[above]{$0$} -- ++(-0.5,0) node[ground](gnd){}

        (B3) node[above]{$0$} -- ++(-0.5,0)  -- (gnd)
        (B1) node[above]{$0$} -- ++(-0.5,0)
        (B0) node[above]{$1$} to[crossing] ++(-1,0) -- (vcc)

        (alb) node[above]{$0$} -- ++(-0.5,0)
        (aeb) node[above]{$1$} to[crossing] ++(-1,0)  -- (vcc)
        
        (ALB) node[right]{Round}
        ;
\end{circuitikz}
\end{center}

\newpage

% Q2
\section{A simple mechanism moves between positions $P=(0000)_2$ and $P=(1111)_2$. Using a magnitude comparator, generate an enable signal, MV\_OK,that is one if the position, P, is 12 or lower or if it is three or higher.}
\begin{table}[H]
    \centering
    \begin{tabular}{c:cccc:l}
        \toprule
        $P_3$&\multicolumn{4}{c}{Threshold}& Relationship\\
        \midrule
        0&0&0&1&0&P valid if greater than threshold\\
        1&1&1&0&1&P valid if less than threshold\\
        &$P_3$&$P_3$&$\overline{P_3}$&$P_3$&\\
        \bottomrule
    \end{tabular}
\end{table}

\,\\

\begin{center}
\begin{circuitikz}
    \getamag{}{(0,0)}
    \tikzset{mux 2x1/.style={muxdemux,muxdemux def={Lh= 5,Rh= 5,NL=9,NR=1,NB=1, w=3},no input leads}}
    \draw
        (4,0) node[mux 2x1](mux){}
        
        (A3) -- ++(-1.5,0) node[circ](A3cntk){} -- ++(-0.5,0) node[left]{$P_3$}
        (A2) -- ++(-1,0) to[crossing] ++(-1,0)  node[left]{$P_2$}
        (A1) -- ++(-1,0) to[crossing] ++(-1,0) node[left]{$P_1$}
        (A0) -- ++(-1,0) to[crossing] ++(-1,0) node[left]{$P_0$}

        (B3) -- ++(-1.5,0) node[circ]{}
        (B2) -- ++(-1.5,0) node[circ]{}
        (B1) node[not port, anchor=out, scale=0.75](not){}
        (not.in) -- (B1-|A3cntk) node[circ]{}
        (B0) -- ++(-1.5,0) node[circ]{}

        (asb) node[above]{$0$} -- ++(-0.5,0) node[ground](gnd){}
        (alb) node[above]{$0$} -- ++(-0.5,0) -- (gnd)
        (aeb) node[above]{$1$} to[crossing] ++(-1,0) node[vcc](vcc){+V}

        (ALB) -- (mux.blpin 2) node[right]{\large$d_0$}
        (ASB) -- (mux.blpin 8) node[right]{\large$d_1$}
        
        (A3cntk) -- ++(0,-8.5) -- (\currcoor-|mux.bbpin 1) -- (mux.bbpin 1) node[above]{\large S}
        (mux.brpin 1) node[left]{Y} -- (mux.rpin 1) node[right]{MV\_OK}
        ;

\end{circuitikz}
\end{center}

\newpage

% Q3
\setcounter{section}{3}
% Q3a
\subsection{Design a magnitude comparator to compare two 4-bit two's complement numbers, X and Y.}
\,\\
\begin{center}
    \begin{circuitikz}
        \getamag{}{(0,0)}
        \draw
            (A2) -- ++(-1.75,0) node[left]{$X_2$}
            (A1) -- ++(-1.75,0) node[left]{$X_1$}
            (A0) -- ++(-1.75,0) node[left]{$X_0$}
            (B2) -- ++(-1.75,0) node[left]{$Y_2$}
            (B1) -- ++(-1.75,0) node[left]{$Y_1$}
            (B0) -- ++(-1.75,0) node[left]{$Y_0$}
            (A3) -- ++(-0.25,0) coordinate(A3r) ++(-1,0) coordinate(A3l) -- ++(-0.5,0) node[left]{$X_3$}
            (B3) -- ++(-0.25,0) coordinate(B3r) ++(-1,0) coordinate(B3l) -- ++(-0.5,0) node[left]{$Y_3$}
            (A3l) -- (B3r)
            (B3l) -- (A3r)

            (asb) node[above]{$0$} -- ++(-0.5,0) node[ground](gnd){}
            (alb) node[above]{$0$} -- ++(-0.5,0) -- (gnd)
            (aeb) node[above]{$1$} to[crossing] ++(-1,0) node[vcc](vcc){+V}
            
            (ALB) node[right]{$X>Y$}
            (AEB) node[right]{$X=Y$}
            (ASB) node[right]{$X<Y$}
            ;
    \end{circuitikz}
\end{center}

\subsection{Complete the following chartgiving the outputs for the corresponding inputs (the leftmost X and Y bits are the MSBs). Indicate whether the output is correct (i.e.“OK”) and give the true relationship of thesignedinputs.}

\begin{table}[H]
    \centering
    \begin{adjustbox}{max width=\textwidth}
    \begin{tabular}{cccc:cccc|cccc:cccc|c:c:c|l}
        \toprule
        \multicolumn{4}{c:}{X (in)}&\multicolumn{4}{c|}{Y (in)}&\multicolumn{4}{c:}{A}&\multicolumn{4}{c|}{B}&\multirow{2}{*}{$X>Y$}&\multirow{2}{*}{$X=Y$}&\multirow{2}{*}{$X<Y$}& OK\\
        \multicolumn{4}{c:}{signed}&\multicolumn{4}{c|}{signed}&\multicolumn{4}{c:}{unsigned}&\multicolumn{4}{c|}{unsigned}&&&&(Signed Result)\\
        \midrule
        1&1&0&0 & 0&1&0&1 & 0&1&0&0 & 1&1&0&1 & 0&0&1 & OK ($-4 < +5$)  \\
        1&0&1&0 & 1&0&0&1 & 1&0&1&0 & 1&0&0&1 & 1&0&0 & OK ($-6 > -7$)  \\
        1&0&0&0 & 0&1&0&1 & 0&0&0&0 & 1&1&0&1 & 0&0&1 & OK ($-8 < +5$)  \\
        1&1&0&1 & 1&1&0&1 & 1&1&0&1 & 1&1&0&1 & 0&1&0 & OK ($-3 = -3$)  \\
        0&0&1&1 & 1&1&0&1 & 1&0&1&1 & 0&1&0&1 & 1&0&0 & OK ($+3 > -5$)  \\
        1&1&0&1 & 1&1&0&0 & 1&1&0&1 & 1&1&0&0 & 1&0&0 & OK ($-3 > -4$)  \\
        0&1&1&1 & 0&1&0&1 & 0&1&1&1 & 0&1&0&1 & 1&0&0 & OK ($+7 > +5$)  \\
        0&0&1&1 & 0&1&0&0 & 0&1&0&0 & 1&1&0&1 & 0&0&1 & OK ($+3 < +4$)  \\
        \bottomrule
    \end{tabular}
    \end{adjustbox}
\end{table}

\newpage

\section{Using magnitude comparators, create a circuit to compare two 8-bit unsigned numbers, X and Y.  Let X connect to the comparator A inputs.}

\begin{center}
    \begin{adjustbox}{max width=\textwidth}
    \begin{circuitikz}[scale=1]
        \getabigfatmag{1}{(0,0)}
        \getabigfatmag{2}{(6,6)}
        \draw
            (1A) node[left]{\Large$X_0-X_3$}
            (2A) node[left]{\Large$X_4-X_7$}

            (1B) node[left]{\Large$Y_0-Y_3$}
            (2B) node[left]{\Large$Y_4-Y_7$}

            (1asb) node[above]{$0$} -- ++(-0.5,0) node[ground](gnd){}
            (1alb) node[above]{$0$} -- ++(-0.5,0) -- (gnd)
            (1aeb) node[above]{$1$} to[crossing] ++(-1,0) node[vcc](vcc){+V}
            
            (1ALB) -- ++(0.5,0) -- (\currcoor|-2alb) -- (2alb)
            (1AEB) -- ++(1  ,0) -- (\currcoor|-2aeb) -- (2aeb)
            (1ASB) -- ++(1.5,0) -- (\currcoor|-2asb) -- (2asb)

            (2ALB) node[right]{\Large$X>Y$}
            (2AEB) node[right]{\Large$X=Y$}
            (2ASB) node[right]{\Large$X<Y$}
            ;
    \end{circuitikz}
    \end{adjustbox}
\end{center}

\end{document}
