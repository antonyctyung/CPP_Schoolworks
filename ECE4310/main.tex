\documentclass{article}
\usepackage{graphicx}
\usepackage[dvipsnames,table]{xcolor}
\usepackage[utf8]{inputenc}
\usepackage{siunitx}
\usepackage[american,siunitx]{circuitikz}
\usepackage{amsmath}
\usepackage{svg}
\usepackage{booktabs}
\usepackage{float}
\usepackage{xparse, xfp}
\usepackage{multirow}
\usepackage{tikz}
\usepackage{karnaugh-map}
\usepackage{pdfpages}
\usepackage{hyperref}
\usepackage{adjustbox}
\usepackage{listings}

\definecolor{codegreen}{rgb}{0,0.6,0}
\definecolor{codegray}{rgb}{0.5,0.5,0.5}
\definecolor{codepurple}{rgb}{0.58,0,0.82}
\definecolor{backcolour}{rgb}{0.95,0.95,0.92}

\lstdefinestyle{mystyle}{
    backgroundcolor=\color{backcolour},   
    commentstyle=\color{codegreen},
    keywordstyle=\color{magenta},
    numberstyle=\tiny\color{codegray},
    stringstyle=\color{codepurple},
    basicstyle=\ttfamily\footnotesize,
    breakatwhitespace=false,         
    breaklines=true,                 
    captionpos=b,                    
    keepspaces=true,                 
    numbers=left,                    
    numbersep=5pt,                  
    showspaces=false,                
    showstringspaces=false,
    showtabs=false,                  
    tabsize=2
}
\lstset{style=mystyle}


\hypersetup{
    colorlinks=true,
    linkcolor=blue,
    filecolor=magenta,      
    urlcolor=cyan,
}
\usetikzlibrary{calc}
%\usepackage[landscape]{geometry}
\renewcommand{\thesubsection}{\thesection.\alph{subsection}}
\newcommand{\equal}{=}
\newcommand{\greyrule}{\arrayrulecolor{black!30}\midrule\arrayrulecolor{black}}
\makeatletter
\newcommand\currcoor{\the\tikz@lastxsaved,\the\tikz@lastysaved}
\makeatother
\newcolumntype{:}{@{\hskip\tabcolsep\color{black!30}\vrule\hskip\tabcolsep}}

\ExplSyntaxOn
\NewExpandableDocumentCommand \groupify { O{\,\allowbreak} m m }
  { \jakob_groupify:nnn {#1} {#2} {#3} }
\cs_new:Npn \jakob_groupify:nnn #1 #2 #3
  { \__jakob_groupify_loop:nnw { 1 } {#2} #3 \q_recursion_tail {#1} \q_recursion_stop }
\cs_new:Npn \__jakob_groupify_loop:nnw #1 #2 #3
  {
    \quark_if_recursion_tail_stop:n {#3}
    \exp_not:n {#3}
    \int_compare:nNnTF {#1} = {#2}
      { \__jakob_groupify_sep:n }
      { \exp_args:Nf \__jakob_groupify_loop:nnw { \int_eval:n { #1+1 } } }
          {#2}
  }
\cs_new:Npn \__jakob_groupify_sep:n #1 #2 \q_recursion_tail #3
  {
    \tl_if_empty:nF {#2} { \exp_not:n {#3} }
    \__jakob_groupify_loop:nnw { 1 } {#1}
    #2 \q_recursion_tail {#3}
  }
\ExplSyntaxOff

\title{ECE 4310\\Operating Systems for Embedded Application\\\,\\Homework 1}
\author{Choi Tim Antony Yung}
\date{February 11, 2021}
\begin{document}
\maketitle

\thispagestyle{empty}
\setcounter{page}{0}

\newpage

\section{(5 pts) What are the three main purposes of an operating system? Explain how the old mainframe computers were different from the computers we have today.}

\subsection*{Main purposes of an operating system:}
\begin{itemize}
  \item Manage hardware
  \item Be an interface to the user 
  \item Manage other processes
\end{itemize}

\subsection*{Mainframes:}
There is no operating system in mainframe computer. It only run a single program ("job") based on some input (e.g. punch cards) and produce some output.

\section{(5 pts) Explain the differences between a program, executable, process, and task.}
A program define the ways of which computer resources are utilized; An executable is a file containing a program that is ready to be loaded into memory and executed; A process is a program loaded into memory and executing; A task is generally a unit of computation on a computer which can be a process or a threat activity.

\newpage

\section{(5 pts) Explain the purpose of system calls within an operating system?  Find two POSIX system calls (using an Internet search) and explain the purpose of the system call, the parameters it accepts, and the return values, if any.}
System calls provide an interface to the services made available by an operating system.

\texttt{chmod} change mode of a file, in other words change the permission certain group of user have on a particular file. It takes the file's path and the mode bits as argument. It return 0 if it was executed successfully and -1 if there were error.

\texttt{chown} change the user and group ownership of a file. It takes the file's path and the user and group id whom it is giving ownership of the file to.  It return 0 if it was executed successfully and -1 if there were error.

\newpage

\section{(10 pts) Research an operating system of your choice, that is not Windows, Linux, or Mac OS. Answerthe following:}
\subsection{What is the name of the OS}
Symbian

\subsection{What company or group of people is responsible for maintaining it}
\begin{itemize}
  \item Symbian Ltd. (1998–2008)
  \item Symbian Foundation (2008–11)
  \item Nokia (2010–11)
  \item Accenture on behalf of Nokia (2011–13)
\end{itemize}


\subsection{When was its first release}
June 5th, 1997

\subsection{How much does it cost}
According to \href{https://web.archive.org/web/20060318221211/http://www.symbian.com/news/pr/2006/pr20063401.html}{an archived version of press release from Symbian}, licence fee starts from \$2.50 per unit. 

\subsection{What type of hardware does it run on}
Smartphone with ARM processors

\subsection{What features set it apart from other operating system}
Support for Qt Framework, built-in webkit browser and multiple language support

\newpage

\section{(5 pts) Using the program below, explain what the output will be at \texttt{LINE A}.}
\lstinputlisting[language=C]{code.c}
\texttt{fork()} will create a replica of the current process with its own copy of address space and states and return 0 to the child process and the process id of the child process to the parent process, therefore the child process will continue at line 14 and the parent at line 18. Since the changes made to value in child process at line 14 affect only the variable in child process' address space, value is unchanged in parent process, therefore it will print value equals to 32 in the parent process. 


\section{(5 pts) Assume that the OS implements Many-to-Many multithreading model.  What is the minimum number of kernel threads required to achieve better concurrency than in the Many-to-One model and why?}
It requires a minimum of two kernel threads for an OS that implements Many-to-Many multithreats model to achieve better concurrency since a user thread can utilize a second kernel thread when another user thread is making a blocking system call.

\section{(10 pts) Explain in detail whether the following problems exhibit data parallelism or task parallelism:}
\subsection{Sum of integers from 1 to \texttt{n}.}
Integers can be distributed to multiple cores all performing the addition operation, thus data parallelism.

\subsection{Find the dot product of two different matrices.}
Dot product can be split into the addition operation and multiplication operation, hence task parallelism.

\subsection{Search of a text file for a string \texttt{s}.}
Text file can be distributed to multiple cores all all performing the search operation, thus data parallelism.

\subsection{Convert the first character of each string in an array to uppercase.}
Array can be distributed to multiple cores all performing the conversion operation, thus data parallelism.

\subsection{Find the product of the integers from 5 to 500 and the average of the integers from 5 to 500.}
It can be split into the finding average operation and multiplication operation, hence task parallelism.

\section{(5 pts) What are the differences between a monolithic kernel and a microkernel?  Explain.}
In a monolithic kernel all functionality are implemented within the same binary file whereas a microkernel hold only the essential functionality and implement other non-essential components as user level programs.

\end{document}