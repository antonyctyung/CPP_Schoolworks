\documentclass{article}
\usepackage[utf8]{inputenc}
\usepackage{siunitx}
\usepackage{graphics}
\usepackage[american,siunitx]{circuitikz}
\usepackage{amsmath}
\usepackage{svg}
\usepackage{booktabs}
\usepackage{float}
\usepackage{xparse, xfp}
%\renewcommand{\thesubsection}{\thesection.\alph{subsection}}
\newcommand{\equal}{=}
\ExplSyntaxOn
\NewDocumentCommand{\defcon}{mm}
 {
  \cs_new:Npx #1 { \fp_eval:n { #2 } }
 }
\ExplSyntaxOff

\title{ECE2101L\\Electrical Circuit Analysis II Laboratory\\\,\\Lab 4\\Response of Source-free Parallel RLC Circuits\\\,\\Prelab\\}
\author{Choi Tim Antony Yung}
\date{24 February 2020}

\begin{document}

\maketitle

\newpage

%1
\section{Second Order RLC Circuit}

\subsection{A second order circuit is a circuit that have two independent energy storage that can be described with a second order differential equation}

\subsection{Below are three different examples of second order circuits.}
\begin{center}
    \begin{circuitikz}
        \draw 
            (0,0) 
            coordinate(R1) -- ++(1.25,0)
            coordinate(n1) -- ++(1.75,0)
            coordinate(L1) -- ++(1.25,0)
            coordinate(n2) -- ++(1.75,0)
            coordinate(C1)
            (R1) to[R=R, i>^=$i_R$] ++(0,-3)
            (n1) ++(0,-0.25) to[open, v^=v] ++(0,-2.5)
            (L1) node[circ,label=v]{} to[L=L, i>^=$i_L$] ++(0,-3) node[ground]{}
            (n2) ++(0,-0.25) to[open, v^=v] ++(0,-2.5)
            (C1) to[C=C, i>^=$i_C$, v=$v_c$] ++(0,-3) --(0,-3)
            ;
    \end{circuitikz}
\end{center}

\begin{center}
    \begin{circuitikz}
        \draw 
            (0,0) 
            coordinate(R1) -- ++(1.25,0)
            coordinate(n1) -- ++(1.75,0)
            coordinate(L1) -- ++(1.25,0)
            coordinate(n2) -- ++(1.75,0)
            coordinate(C1)
            (R1) to[R=$R_1$] ++(0,-3)
            (L1) to[C=$C_1$] ++(0,-3) 
            (C1) to[R=$R_2$] ++(0,-2) to[C=$C_2$] ++(0,-1) --(0,-3)
            ;
    \end{circuitikz}
\end{center}

\begin{center}
    \begin{circuitikz}
        \draw 
            (0,0) 
            coordinate(R1) -- ++(1.25,0)
            coordinate(n1) -- ++(1.75,0)
            coordinate(L1) -- ++(1.25,0)
            coordinate(n2) -- ++(1.75,0)
            coordinate(C1)
            (R1) to[R=$R_1$] ++(0,-4)
            (L1) to[L=$L_1$] ++(0,-4) 
            (C1) to[R=$R_2$] ++(0,-2) to[L=$L_2$] ++(0,-2) --(0,-4)
            ;
    \end{circuitikz}
\end{center}

\pagebreak

%2
\section{Source-free Second Order Parallel RLC Circuits }
\begin{center}
    \begin{circuitikz}
        \draw 
            (0,0) 
            coordinate(R1) -- ++(1.25,0)
            coordinate(n1) -- ++(1.75,0)
            coordinate(L1) -- ++(1.25,0)
            coordinate(n2) -- ++(1.75,0)
            coordinate(C1)
            (R1) to[R=R, i>^=$i_R$] ++(0,-3)
            (n1) ++(0,-0.25) to[open, v^=v] ++(0,-2.5)
            (L1) node[circ,label=v]{} to[L=L, i>^=$i_L$] ++(0,-3) node[ground]{}
            (n2) ++(0,-0.25) to[open, v^=v] ++(0,-2.5)
            (C1) to[C=C, i>^=$i_C$, v=$v_c$] ++(0,-3) --(0,-3)
            ;
    \end{circuitikz}
\end{center}
$$\frac{d^2v}{dt^2}+\frac{1}{RC}\frac{dv}{dt}+\frac{1}{LC}v=0$$

\subsection{The characteristic equation in relation to the above differential equation is the following:}
$$s^2+\frac{1}{RC}s+\frac{1}{LC}=0$$

\subsection{The formulae for calculating damping factor $\alpha$ and undamped natural frequency $\omega_0$ for the above parallel RLC circuit is the following:}
$$\alpha = \frac{1}{2RC}$$
$$\omega_0 = \sqrt{\frac{1}{LC}}$$

\subsection{The expressions for determining the roots of the characteristic equation in terms of RLC and in terms of $\alpha$ and $\omega_0$ is the following:}
$$-\alpha \pm\sqrt{\alpha^2-\omega_0^2}$$

\subsection{The condition for solution in underdamped, critically damped, and overdamped case in terms of $\alpha$ and $\omega_0$ is the following:}
\[ \begin{cases} 
      \alpha < \omega_0 & underdamped \\
      \alpha = \omega_0 & critically\; damped \\
      \alpha > \omega_0 & overdamped 
   \end{cases}
\]

\subsection{The solution of v(t) for underdamped and overdamped cases are as follows:}
\[ 
v(t)=
    \begin{cases} 
        e^{-\alpha t}[B_1cos(\sqrt{\omega_0^2-\alpha^2}t)+B_2sin(\sqrt{\omega_0^2-\alpha^2}t)] & underdamped \\
        A_1e^{(-\alpha+\sqrt{\alpha^2-\omega_0^2})t} + A_2e^{(-\alpha-\sqrt{\alpha^2-\omega_0^2})t} & overdamped 
    \end{cases}
\]

\subsection{The coefficient $A_1$ and $A_2$ can be calculated based on initial condition $v(0)$ and $v'(0)$ as follows:}
For the underdamped case:
$$v(0) =e^{-\alpha 0}[B_1cos(\sqrt{\omega_0^2-\alpha^2}0)+B_2sin(\sqrt{\omega_0^2-\alpha^2}0)] = B_1$$
$$v'(0) =e^{-\alpha 0}[(-\alpha B_1 + \sqrt{\omega_0^2-\alpha^2} B_2)cos(\sqrt{\omega_0^2-\alpha^2}0)+(-\sqrt{\omega_0^2-\alpha^2}B_1-\alpha B_2)sin(\sqrt{\omega_0^2-\alpha^2}0)]$$
$$v'(0)=-\alpha B_1 + \sqrt{\omega_0^2-\alpha^2} B_2$$
$$ B_2 = \frac{v'(0) + \alpha v(0)}{\beta} $$

For the overdamped case:
$$v(0) = A_1e^{(-\alpha+\sqrt{\alpha^2-\omega_0^2})0} + A_2e^{(-\alpha-\sqrt{\alpha^2-\omega_0^2})0} = A_1 + A_2$$
$$v'(0) = A_1(-\alpha+\sqrt{\alpha^2-\omega_0^2})e^{(-\alpha+\sqrt{\alpha^2-\omega_0^2})0} + A_2(-\alpha-\sqrt{\alpha^2-\omega_0^2})e^{(-\alpha-\sqrt{\alpha^2-\omega_0^2})0}$$
$$v'(0) = A_1(-\alpha+\sqrt{\alpha^2-\omega_0^2}) + A_2(-\alpha-\sqrt{\alpha^2-\omega_0^2})$$
$$A_1 = \frac{v'(0)-v(0)\sqrt{\alpha^2-\omega_0^2}}{\sqrt{\alpha^2+\omega_0^2}-\sqrt{\alpha^2-\omega_0^2}}$$
$$A_2 = \frac{v'(0)-v(0)\sqrt{\alpha^2+\omega_0^2}}{\sqrt{\alpha^2-\omega_0^2}-\sqrt{\alpha^2+\omega_0^2}}$$

\end{document}