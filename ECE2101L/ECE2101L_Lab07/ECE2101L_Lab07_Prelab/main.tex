\documentclass{article}
\usepackage[utf8]{inputenc}
\usepackage{siunitx}
\usepackage{graphics}
\usepackage[american,siunitx]{circuitikz}
\usepackage{amsmath}
\usepackage{svg}
\usepackage{booktabs}
\usepackage{float}
\usepackage{xparse, xfp}
\usepackage{graphicx} 
\usepackage{steinmetz}
%\renewcommand{\thesubsection}{\thesection.\alph{subsection}}
\newcommand{\equal}{=}
\ExplSyntaxOn
\NewDocumentCommand{\defcon}{mm}
 {
  \cs_new:Npx #1 { \fp_eval:n { #2 } }
 }
\ExplSyntaxOff

\title{ECE2101L\\Electrical Circuit Analysis II Laboratory\\\,\\Lab 7\\Input and Output Impedances of AC Black Boxes\\\,\\Prelab\\}
\author{Choi Tim Antony Yung}
%\author{Choi Tim Antony Yung\\\,\\Willis Nguyen\\Phineas Cozmiuc}
\date{16 March 2020}

\begin{document}

\clearpage\maketitle
\thispagestyle{empty}
\newpage
\setcounter{page}{1}

\section{Thevenin and Norton equivalent circuits}
\subsection{Simplification of circuit between two node}
Thevenin equivalent circuit is a simplification of a circuit between two node into a relatively simple circuit consist of a voltage source and an impedance in series, while Norton equivalent circuit simplifies into a current source and an impedance in parallel source.\\

Thevenin equivalent circuit:\\
\begin{center}
\begin{circuitikz}
    \draw
    (0,0) to[voltage source, l_=$V_{TH}$](0,-2) -- ++ (3,0) node[circ,label = b]{}
    (0,0) to[generic, label = $Z_{TH}$] (3,0) node[circ,label = a]{}
    
    ;
\end{circuitikz}
\end{center}

Norton equivalent circuit:\\
\begin{center}
\begin{circuitikz}
    \draw
    (0,-2) to[current source, label=$I_{N}$](0,0) -- (3,0) node[circ,label = a]{}
    (1.5,0) to[generic, label = $Z_{N}$] ++(0,-2)
    (0,-2) -- ++ (3,0) node[circ,label = b]{}
    ;
\end{circuitikz}
\end{center}

\subsection{$Z_N=Z_{TH}=\frac{V_{TH}}{I_{TH}}$}

\section{Input impedance}
\subsection{$Z=R_1+\frac{1}{\frac{1}{R_2}+\frac{1}{\frac{1}{j\omega C}}}$}
\subsection{Magnitude and phase of Z can be determined as follow}
$$Z=R_1+\frac{1}{\frac{1}{R_2}+\frac{1}{\frac{1}{j\omega C}}}=R_1+\frac{R_2}{1+R_2j\omega C}=R_1+\frac{R_2(1-R_2j\omega C)}{1+(R_2\omega C)^2}=R_1+\frac{R_2-R_2^2j\omega C}{1+(R_2\omega C)^2}$$
$$Z=(R_1+\frac{R_2}{1+(R_2\omega C)^2})-\frac{R_2^2\omega C}{1+(R_2\omega C)^2}j$$
$$Z=\sqrt{(R_1+\frac{R_2}{1+(R_2\omega C)^2})^2+(\frac{R_2^2\omega C}{1+(R_2\omega C)^2})^2}\phase{tan^{-1}\frac{-\frac{R_2^2\omega C}{1+(R_2\omega C)^2}}{R_1+\frac{R_2}{1+(R_2\omega C)^2}}}$$

\section{AC circuit analysis}
\subsection{The impedance of the capacitor $Z_c$ and capacitance C can be found as follow:}
$$V=V_{R_1}+V_0=V_{R_1}+4.8015\phase{-27.74^{\circ}}=7\phase{0^{\circ}}$$
$$V_{R_1}=7\phase{0^{\circ}}-4.8015\phase{-27.74^{\circ}}=3.5439\phase{39.10^{\circ}}\,V$$
$$I=\frac{V_{R_1}}{R_1}=\frac{3.5439{\phase{39.10^{\circ}}}}{180}=0.01969\phase{39.10^{\circ}}\,A$$
$$I=I_{R_2}+I_{C}$$
$$I_{R_2}=\frac{V_{R_2}}{R_2}=\frac{V_0}{R_2}$$
$$I_C=0.01969\phase{39.10^{\circ}}-\frac{4.8015\phase{-27.74^{\circ}}}{620}=0.01810\phase{62.26^{\circ}}\,A$$
$$Z_C=\frac{V_{C}}{I_{C}}=\frac{V_{0}}{I_{C}}=\frac{4.8015\phase{-27.74^{\circ}}}{0.01810\phase{62.26^{\circ}}}\approx265.3\phase{-90^{\circ}}=-265.3j\,\Omega$$
$$C=\frac{-j}{\omega Z_C}=\frac{-j}{(2\pi 60)(-265.3j)}\approx\SI{10}{\micro\farad}$$
\subsection{Phase difference}
The input current I is $19.69\phase{39.10^{\circ}}$\,mA with reference to $V=7\phase{0^{\circ}}$\,V. Therefore, the input current leads the input voltage by a phase difference of $39.1^{\circ}$

\end{document}