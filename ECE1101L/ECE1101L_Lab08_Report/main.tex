\documentclass{article}
\usepackage[utf8]{inputenc}
\usepackage{siunitx}
\usepackage[american,siunitx]{circuitikz}
\usepackage{amsmath}
\usepackage{svg}
\usepackage{booktabs}
\usepackage{float}
\renewcommand{\thesubsection}{\thesection.\alph{subsection}}
\newcommand{\equal}{=}

\title{ECE1101L - LAB 8\\EQUIVALENT RESISTANCES}
\author{Choi Tim Antony Yung}
\date{17 October 2019}

\begin{document}

\maketitle

\section*{Objective}
The objective of this lab is to measure and calculate equivalent resistance of different configuration of series or parallel resistors.

\section*{Lab Result}
Lab Partner: Anthony Nursalim

%1
\section{A circuit as illustrated below was set up with a varying voltage source}
\begin{center}
    \begin{circuitikz}
        \draw 
            (0,2) 
            to[V_=$V$] (0,0)
            (0,2) to[short,i>^=I] (1,2) to[R=\SI{2}{\kilo\ohm}] (3,2)--(5,2)
            (3,2)to[R=\SI{4.7}{\kilo\ohm}] (3,0)
            (5,2)to[R=\SI{10}{\kilo\ohm}] (5,0)--(0,0)
            %to[R=\SI{4.7}{\kilo\ohm},v=$V_1$]
            ;
    \end{circuitikz}
\end{center}

%1a
\subsection{The resistance of the three resistors in circuit were measured}
The \SI{2}{\kilo\ohm} resistor was measured to have a resistance of \SI{1.984}{\kilo\ohm}, the \SI{4.7}{\kilo\ohm} resistor was measured to have a resistance of \SI{4.63}{\kilo\ohm}, and the \SI{10}{\kilo\ohm} resistor was measured to have a resistance of \SI{9.96}{\kilo\ohm}.
\begin{table}[H]
\centering
    \begin{tabular}{@{}r r c@{}}
         \toprule
         measured & nominal & \%diff  \\
         \midrule
         \SI{1.984}{\kilo\ohm} & \SI{2}{\kilo\ohm} & -0.8\% \\
         \SI{4.63}{\kilo\ohm} & \SI{4.7}{\kilo\ohm} & -1.5\% \\
         \SI{9.96}{\kilo\ohm} & \SI{10}{\kilo\ohm} & -0.4\% \\
         \bottomrule
    \end{tabular}
\end{table}

%1b
\subsection{Current I was measured as voltage V was changed}
Below is the data set obtained and the chart generated from it.
\begin{center}
    \begin{tabular}{c c}
         \toprule
         Current (A) & Voltage (V)\\
         \midrule
         {$\!\begin{aligned}
             -950&\times 10^{-6} \\
             -744&\times 10^{-6} \\
             -580&\times 10^{-6} \\
             -474&\times 10^{-6} \\
             -378&\times 10^{-6} \\
             -187&\times 10^{-6} \\
             -0.6&\times 10^{-6} \\
             188&\times 10^{-6} \\
             372&\times 10^{-6} \\
             475&\times 10^{-6} \\
             571&\times 10^{-6} \\
             766&\times 10^{-6} \\
             950&\times 10^{-6} 
         \end{aligned}$} & {$\!\begin{aligned}
            -&4.99 \\
            -&3.92 \\
            -&3.05 \\
            -&2.49 \\
            -&1.98 \\
            -&0.987 \\
            -&0.003 \\
            &0.992 \\
            &1.967 \\
            &2.50 \\
            &3.00 \\
            &4.02 \\
            &5.00 \\
         \end{aligned}$}\\
         \bottomrule
    \end{tabular}
\end{center}
\begin{figure}[h]
\includesvg[width = 345pt]{chart}
\end{figure}
From the slope of the above chart, we can derive the equivalent resistance $R_{EQ} = \SI{5258}{\ohm}$. Thus, the relation between V in volts and I in amperes is $V = 5258\cdot I$.

%1c
\subsection{The equivalent resistance of the resistors circuit was measured}
Using an ohmmeter in place of the voltage source, $R_{EQ}$ was measured to be \SI{5140}{\ohm}.

%1d
\subsection{The measured equivalent resistance is lower than the equivalent resistance derived from the above chart}
\begin{table}[H]
\centering
    \begin{tabular}{@{} l r r r@{}}
         \toprule
         &measured & derived & \%diff  \\
         \midrule
         $R_{EQ}$ &\SI{5140}{\ohm} & \SI{5258}{\ohm} & -2.24\% \\
         \bottomrule
    \end{tabular}
\end{table}

%1e
\subsection{With an equivalent resistance of 5140 and a 2.5 voltage source, I was predicted to be \SI{486}{\micro\ampere}}
\begin{center}
    \begin{circuitikz}
        \draw 
            (0,2) 
            to[V_=$V\equal\SI{2.5}{\volt}$] (0,0)
            (0,2) to[short,i>^=I] (2,2) to[R=\SI{5140}{\ohm}] (2,0)--(0,0)
            ;
    \end{circuitikz}
\end{center}
In the above circuit, we can, using the Ohm's law, predict I to be $\frac{V}{R}=\frac{2.5}{5140}=486\times10^{-6}A=\SI{486}{\micro\ampere}$

%1f
\subsection{I was measured with a 2.5V voltage source}
I was measured to be \SI{475}{\micro\ampere}.

%1g
\subsection{The measured current I is lower than the current I derived}
\begin{table}[H]
\centering
    \begin{tabular}{@{} l r r r@{}}
         \toprule
         &measured & derived & \%diff  \\
         \midrule
         $I$ &\SI{475}{\micro\ampere} & \SI{486}{\micro\ampere} & -2.34\% \\
         \bottomrule
    \end{tabular}
\end{table}

%2
\section{A circuit as illustrated below was set up without a voltage source}
\begin{center}
    \begin{circuitikz}
        \draw 
            (0,2) 
            to[open, v^=V] (0,0) -- (6,0)
            (0,2)
            (0,2) to[short,i>^=I] (1,2) to[R=$R_1$\equal\SI{4.7}{\kilo\ohm}] (1,0)
            (1,2) -- (3.5,2)
            to[R=$R_2$\equal\SI{10}{\kilo\ohm}] (3.5,0)
            (3.5,2) -- (6,2)
            to[R=$R_3$\equal\SI{10}{\kilo\ohm}] (6,0)
            ;
    \end{circuitikz}
\end{center}


%2a
\subsection{The resistance of the three resistors in circuit were measured}
The \SI{4.7}{\kilo\ohm} resistor was measured to have a resistance of \SI{4.63}{\kilo\ohm}, the first \SI{10}{\kilo\ohm} resistor was measured to have a resistance of \SI{9.96}{\kilo\ohm}, and the second \SI{10}{\kilo\ohm} resistor was measured to have a resistance of \SI{9.97}{\kilo\ohm}.
\begin{table}[H]
\centering
    \begin{tabular}{@{}l r r r@{}}
         \toprule
         &measured & nominal & \%diff  \\
         \midrule
         $R_1$&\SI{4.63}{\kilo\ohm} & \SI{4.7}{\kilo\ohm} & -1.5\% \\
         $R_2$&\SI{9.96}{\kilo\ohm} & \SI{10}{\kilo\ohm} & -0.4\% \\
         $R_3$&\SI{9.97}{\kilo\ohm} & \SI{10}{\kilo\ohm} & -0.3\% \\
         \bottomrule
    \end{tabular}
\end{table}

%2b
\subsection{The resistance across the open circuit is measured}
The equivalent resistance of the parallel resistors circuit was measured to be $R_{EQ}=\SI{2400}{\ohm}$.

%2c
\subsection{The equivalent resistance of the parallel resistors circuit was calculated below}
\begin{center}
    $R_{EQ}=\frac{1}{\frac{1}{4630}+\frac{1}{9960}+\frac{1}{9970}}=\SI{2400}{\ohm}$
\end{center}

%2d
\subsection{The calculated value and the measured value of the equivalent resistance of the parallel resistors circuit are the same}
\begin{table}[H]
\centering
    \begin{tabular}{@{} l r r r@{}}
         \toprule
         &measured & calculated & \%diff  \\
         \midrule
         $R_{EQ}$ &\SI{2400}{\ohm} & \SI{2400}{\ohm} & 0\% \\
         \bottomrule
    \end{tabular}
\end{table}

%3
\section{A circuit as illustrated below was set up with a 5.00V voltage source}
\begin{center}
    \begin{circuitikz}
        \draw 
            (0,3) 
            to[V_=$V\equal\SI{5}{\volt}$] (0,0) -- (6,0)
            (0,3)
            (0,3) to[short,i>^=I] (1,3) to[R=$R_1$\equal\SI{4.7}{\kilo\ohm}] (3.5,3)
            to[short, i=$I_1$] (3.5,2) to[R=$R_2$\equal\SI{10}{\kilo\ohm}] (3.5,0)
            (3.5,3) -- (6,3)
            -- (6,0)
            ;
    \end{circuitikz}
\end{center}

%3a
\subsection{The current $I_1$ was measured}
The current $I_1$ was measured to be \SI{0}{\ampere}.

%3b
\subsection{There are no epc flowing through $R_2$}
As epc flow through where there is least resistance, as they arrive at the node connecting $R_1$ and $R_2$, they chose to flow through the short circuit, as it, having an ideal resistance of zero, have far less resistance than $R_2$. 

%3c
\subsection{$R_{EQ} = R_1$}
As there are no current flowing through $R_2$, $R_1$ is de facto the only resistor in the circuit. Therefore, the equivalent resistance of the resistors circuit is $R_1=\SI{4.63}{\kilo\ohm}$.

%3d
\subsection{The equivalent resistance of the resistors circuit is measured}
The equivalent resistance of the resistors circuit was measured to be $R_{EQ}=\SI{4.63}{\kilo\ohm}$.

%3e
\subsection{The calculated value and the measured value of the equivalent resistance of the resistors circuit are the same}
\begin{table}[H]
\centering
    \begin{tabular}{@{} l r r r@{}}
         \toprule
         &measured & calculated & \%diff  \\
         \midrule
         $R_{EQ}$ &\SI{4630}{\ohm} & \SI{4630}{\ohm} & 0\% \\
         \bottomrule
    \end{tabular}
\end{table}

%4
\section{A circuit as illustrated below was set up with a 5.00V voltage source}
\begin{center}
    \begin{circuitikz}
        \draw 
            (0,2) 
            to[V_=$V\equal\SI{5}{\volt}$] (0,0) -- (7,0)
            (0,2)
            (0,2) to[short,i>^=I] (1,2) to[R=$R_1$\equal\SI{4.7}{\kilo\ohm}] (3.5,2)
            to[R,v^=$V_1$,l_=$R_2$\equal\SI{10}{\kilo\ohm}] (3.5,0)
            (3.5,2) to [short, i=$I_2$] (5,2) to [R=$R_3$\equal\SI{10}{\kilo\ohm}] (7,2)
            to[open, v=$V_2$] (7,0)
            ;
    \end{circuitikz}
\end{center}

%4a
\subsection{The current $I_2$, the voltage $V_1$ and $V_2$ was measured}
The current $I_2$ was measured to be \SI{0}{\ampere}, the voltage $V_1$ was measured to be \SI{3.39}{\volt} and the voltage $V_2$ was measured to be the same, \SI{3.39}{\volt}.

%4b
\subsection{There are no current flowing through $R_3$, thus the voltage across $R_3$ was the same}
As the current flow through where there is least resistance, as they arrive at the node connecting $R_1$ and $R_2$, they chose to flow through $R_2$, as it have far less resistance than an open circuit, having a theoretically infinite resistance. As no current is flowing through $R_3$, the $R_3$ will be zero as well by Ohm's Law, meaning there are no potential difference before and after $R_3$, thus the voltage different across $R_2$ is the same as the one across the open circuit.

%4c
\subsection{$R_{EQ} = R_1+R_2$ }
As there are no current flowing through $R_3$, it is, in effect, a series circuit of $R_1$ and $R_2$. Therefore, the equivalent resistance of the resistors circuit is $R_1+R_2 = 4630+9960=\SI{14590}{\ohm}$.
 
%4d
\subsection{The equivalent resistance of the resistors circuit is measured}
The equivalent resistance of the resistors circuit was measured to be $R_{EQ}=\SI{1.459}{\kilo\ohm}$.

%4e
\subsection{The calculated value and the measured value of the equivalent resistance of the resistors circuit are the same}
\begin{table}[H]
\centering
    \begin{tabular}{@{} l r r r@{}}
         \toprule
         &measured & calculated & \%diff  \\
         \midrule
         $R_{EQ}$ &\SI{14590}{\ohm} & \SI{14590}{\ohm} & 0\% \\
         \bottomrule
    \end{tabular}
\end{table}


\end{document}
