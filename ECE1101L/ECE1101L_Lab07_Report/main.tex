\documentclass{article}
\usepackage[utf8]{inputenc}
\usepackage{siunitx}
\usepackage[american,siunitx]{circuitikz}
\renewcommand{\thesubsection}{\thesection.\alph{subsection}}
\newcommand{\equal}{=}

\title{ECE1101L - LAB 7\\SERIES RESISTOR CIRCUITS}
\author{Choi Tim Antony Yung}
\date{17 October 2019}

\begin{document}

\maketitle

\section*{Objective}
The objective of this lab is to explore the behavior of a series resistor circuit

\section*{Lab Result}
Lab Partner: Anthony Nursalim

%1
\section{A circuit as illustrated below was set up with a \SI{8.00}{\volt} voltage source}
\begin{center}
    \begin{circuitikz}
        \draw 
            (-1,3) 
            to[V_=$V_s\equal\SI{8}{\volt}$] (-1,0) -- (6,0)
            (-1,3) to[short,i=I](0,3)
            (6,3) to[R=\SI{10}{\kilo\ohm}, v=$V_2$] (6,0)
            (0,3) to[R=\SI{4.7}{\kilo\ohm},v=$V_1$] (3,3)
            -- (6,3)
            ;
    \end{circuitikz}
\end{center}

%1a
\subsection{The resistance of the two resistors in circuit were measured}
The \SI{4.7}{\kilo\ohm} resistor was measured to have a resistance of \SI{4.73}{\kilo\ohm} while the \SI{10}{\kilo\ohm} resistor was measured to have a resistance of \SI{9.97}{\kilo\ohm}.
\begin{center}
    \begin{tabular}{|c|c c c|}
         \hline
         & measured & nominal & \%diff  \\
         \hline
         $R_1$ & \SI{4.73}{\kilo\ohm} & \SI{4.7}{\kilo\ohm} & 0.6\% \\
         $R_2$ & \SI{9.97}{\kilo\ohm} & \SI{10}{\kilo\ohm} & -0.3\% \\
         \hline
    \end{tabular}
\end{center}

%1b
\subsection{The current flowing through all circuit elements in this circuit was measured}
The current, I, was measured to be the same: \SI{0.539}{\milli\ampere} across all circuit elements.

%1c
\subsection{Prelab: The \SI{10}{\kilo\ohm} resistor should have the larger magnitude voltage}
As both resistors have the same current I flowing through, by Ohm's law, with current held constant, voltage would be directly proportional to resistance, therefore the \SI{10}{\kilo\ohm} resistor ($R_2$) should have the higher magnitude voltage.

%1d
\subsection{The voltage across each resistor were measured}
$V_1$ was measured to be \SI{2.57}{\volt} and $V_2$ was measured to be \SI{5.42}{\volt}. As conjectured above, $R_2$ have the higher magnitude voltage.

%1e
\subsection{KVL was verified within margin of error}
KVL, when applied to this circuit, yield the following equation:
\[V=V_1+V_2\]
When substituted with measured values, it became:
\[\SI{8.00}{\volt}\approx\SI{7.99}{\volt}=\SI{2.57}{\volt}+\SI{5.42}{\volt}\]
The above is true within margin of error, thus KVL was verified.

%1f
\subsection{$V_1$ and $V_2$ can be calculated with Ohm's Law: $V=I\cdot R$}
\[ V_1 = I\cdot R_1 = \SI{0.539}{\milli\ampere}\cdot\SI{4.73}{\kilo\ohm}=\SI{2.55}{\volt} \]
\[ V_2 = I\cdot R_2 = \SI{0.539}{\milli\ampere}\cdot\SI{9.97}{\kilo\ohm}=\SI{5.37}{\volt} \]

%1g
\subsection{$V_1$ and $V_2$ were measured and calculated}
\begin{center}
    \begin{tabular}{|c|c c c|}
         \hline
         & measured & calculated & \%diff  \\
         \hline
         $V_1$ & \SI{2.57}{\volt} & \SI{2.55}{\volt} & 0.8\% \\
         $V_1$ & \SI{5.42}{\volt} & \SI{5.37}{\volt} & 0.9\% \\
         \hline
    \end{tabular}
\end{center}

%2
\section{A circuit as illustrated below was set up with a 8.00 V voltage source}
\begin{center}
    \begin{circuitikz}
        \draw 
            (-1,3) 
            to[V_=$V_s\equal\SI{8}{\volt}$] (-1,0) -- (6,0)
            (-1,3) to[short,i=I](0,3)
            (6,3)--(6,2.5) to[R=\SI{10}{\kilo\ohm}, v=$V_3$] (6,0.5)--(6,0)
            (0,3) to[R=\SI{4.7}{\kilo\ohm},v=$V_1$] (3,3)
            (3,3) to[R=\SI{10}{\kilo\ohm}, v=$V_2$] (6,3)
            ;
    \end{circuitikz}
\end{center}

%2a
\subsection{The resistance of the three resistors in circuit were measured}
The \SI{4.7}{\kilo\ohm} resistor was measured to have a resistance of \SI{4.73}{\kilo\ohm}, the first \SI{10}{\kilo\ohm} resistor ($R_1$) was measured to have a resistance of \SI{9.97}{\kilo\ohm}, and the second \SI{10}{\kilo\ohm} resistor ($R_2$) was measured to have a resistance of \SI{9.95}{\kilo\ohm} .
\begin{center}
    \begin{tabular}{|c|c c c|}
         \hline
         & measured & nominal & \%diff  \\
         \hline
         $R_1$ & \SI{4.73}{\kilo\ohm} & \SI{4.7}{\kilo\ohm} & 0.6\% \\
         $R_2$ & \SI{9.97}{\kilo\ohm} & \SI{10}{\kilo\ohm} & -0.3\% \\
         $R_3$ & \SI{9.95}{\kilo\ohm} & \SI{10}{\kilo\ohm} & -0.5\% \\
         \hline
    \end{tabular}
\end{center}

%2b
\subsection{The current flowing through all circuit elements in this circuit was measured}
The current, I, was measured to be the same: \SI{0.322}{\milli\ampere} across all circuit elements.

%2c
\subsection{Prelab: The two \SI{10}{\kilo\ohm} resistor should have the larger magnitude voltage}
As all three resistors have the same current I flowing through, by Ohm's law, with current held constant, voltage would be directly proportional to resistance, therefore the \SI{10}{\kilo\ohm} resistors should have the higher magnitude voltage. More specifically, as $R_2$ was measured to have the largest resistance among the three resistors, it should have the largest magnitude voltage.

%2d
\subsection{The voltage across each resistor were measured}
$V_1$ was measured to be \SI{1.537}{\volt}, $V_2$ was measured to be \SI{3.23}{\volt}, and $V_3$ was measured to be \SI{3.22}{\volt}. As conjectured above, $R_2$ have the highest magnitude voltage.

%2e
\subsection{KVL was verified within margin of error}
KVL, when applied to this circuit, yield the following equation:
\[V=V_1+V_2+V_3\]
When substituted with measured values, it became:
\[\SI{8.00}{\volt}\approx\SI{7.99}{\volt}=\SI{1.537}{\volt}+\SI{3.23}{\volt}+\SI{3.22}{\volt}\]
The above is true within margin of error, thus KVL was verified.

%2f
\subsection{$V_1$ and $V_2$ can be calculated with Ohm's Law: $V=I\cdot R$}
\[ V_1 = I\cdot R_1 = \SI{0.322}{\milli\ampere}\cdot\SI{4.73}{\kilo\ohm}=\SI{1.52}{\volt} \]
\[ V_2 = I\cdot R_2 = \SI{0.322}{\milli\ampere}\cdot\SI{9.97}{\kilo\ohm}=\SI{3.21}{\volt} \]
\[ V_3 = I\cdot R_3 = \SI{0.322}{\milli\ampere}\cdot\SI{9.95}{\kilo\ohm}=\SI{3.20}{\volt} \]

%2g
\subsection{$V_1$, $V_2$, and $V_3$ were measured and calculated}
\begin{center}
    \begin{tabular}{|c|c c c|}
         \hline
         & measured & calculated & \%diff  \\
         \hline
         $V_1$ & \SI{1.537}{\volt} & \SI{1.52}{\volt} & 1\% \\
         $V_1$ & \SI{3.23}{\volt} & \SI{3.21}{\volt} & 0.9\% \\
         $V_1$ & \SI{3.22}{\volt} & \SI{3.20}{\volt} & 0.9\% \\
         \hline
    \end{tabular}
\end{center}

%3
\section{The MATLAB program and the plot are attached}

\end{document}
