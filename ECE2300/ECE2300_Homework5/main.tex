\documentclass{article}
\usepackage[table]{xcolor}
\usepackage[utf8]{inputenc}
\usepackage{siunitx}
\usepackage[american,siunitx]{circuitikz}
\usepackage{amsmath}
\usepackage{svg}
\usepackage{booktabs}
\usepackage{float}
\usepackage{xparse, xfp}
%\usepackage[landscape]{geometry}
\renewcommand{\thesubsection}{\thesection.\alph{subsection}}
\newcommand{\equal}{=}
\newcommand{\greyrule}{\arrayrulecolor{black!30}\midrule\arrayrulecolor{black}}
\newcolumntype{:}{@{\hskip\tabcolsep\color{black!30}\vrule\hskip\tabcolsep}}

\ExplSyntaxOn
\NewExpandableDocumentCommand \groupify { O{\,\allowbreak} m m }
  { \jakob_groupify:nnn {#1} {#2} {#3} }
\cs_new:Npn \jakob_groupify:nnn #1 #2 #3
  { \__jakob_groupify_loop:nnw { 1 } {#2} #3 \q_recursion_tail {#1} \q_recursion_stop }
\cs_new:Npn \__jakob_groupify_loop:nnw #1 #2 #3
  {
    \quark_if_recursion_tail_stop:n {#3}
    \exp_not:n {#3}
    \int_compare:nNnTF {#1} = {#2}
      { \__jakob_groupify_sep:n }
      { \exp_args:Nf \__jakob_groupify_loop:nnw { \int_eval:n { #1+1 } } }
          {#2}
  }
\cs_new:Npn \__jakob_groupify_sep:n #1 #2 \q_recursion_tail #3
  {
    \tl_if_empty:nF {#2} { \exp_not:n {#3} }
    \__jakob_groupify_loop:nnw { 1 } {#1}
    #2 \q_recursion_tail {#3}
  }
\ExplSyntaxOff

\title{ECE 2300\\Digital Logic Design\\\,\\Homework 5}
\author{Choi Tim Antony Yung}
\begin{document}
\maketitle

\thispagestyle{empty}
\setcounter{page}{0}

\newpage

\section{Using an 8:1 mux, create a circuit to generate}
$$f(WXYZ)=\overline{W}\cdot\overline{Z}+\overline{X}\cdot\overline{Y}\cdot Z+X\cdot Y\cdot Z$$

\begin{table}[H]
    \centering
    \begin{tabular}{ccc:c|c}
        \toprule
        W&X&Y&Z&$f$\\
        \midrule
        0&0&0&0&1\\
        0&0&0&1&1\\
        \greyrule
        0&0&1&0&1\\
        0&0&1&1&0\\
        \greyrule
        0&1&0&0&1\\
        0&1&0&1&0\\
        \greyrule
        0&1&1&0&1\\
        0&1&1&1&1\\
        \bottomrule
    \end{tabular}
    \quad
    \begin{tabular}{ccc:c|c}
        \toprule
        W&X&Y&Z&$f$\\
        \midrule
        1&0&0&0&0\\
        1&0&0&1&1\\
        \greyrule
        1&0&1&0&0\\
        1&0&1&1&0\\
        \greyrule
        1&1&0&0&0\\
        1&1&0&1&0\\
        \greyrule
        1&1&1&0&0\\
        1&1&1&1&1\\
        \bottomrule
    \end{tabular}
\end{table}
\,\\
\begin{center}
    \begin{circuitikz}
        \tikzset{mux/.style={muxdemux,muxdemux def={Lh= 10,Rh= 7,NL=10},no input leads}}
        \draw
        (0,0) node[mux](mux){}

        (mux.blpin 2) node[right]{$d_0$} -- (mux.lpin 2) coordinate (d0)
        (mux.blpin 3) node[right]{$d_1$} -- (mux.lpin 3) coordinate (d1)
        (mux.blpin 4) node[right]{$d_2$} -- (mux.lpin 4) coordinate (d2)
        (mux.blpin 5) node[right]{$d_3$} -- (mux.lpin 5) coordinate (d3)
        (mux.blpin 6) node[right]{$d_4$} -- (mux.lpin 6) coordinate (d4)
        (mux.blpin 7) node[right]{$d_5$} -- (mux.lpin 7) coordinate (d5)
        (mux.blpin 8) node[right]{$d_6$} -- (mux.lpin 8) coordinate (d6)
        (mux.blpin 9) node[right]{$d_7$} -- (mux.lpin 9) coordinate (d7)

        (mux.bbpin 1) node[above]{\small{$S_2$}} -- (mux.bpin 1) coordinate (S2)
        (mux.bbpin 2) node[above]{\small{$S_1$}} -- (mux.bpin 2) coordinate (S1)
        (mux.bbpin 3) node[above]{\small{$S_0$}} -- (mux.bpin 3) coordinate (S0)

        (mux.brpin 1) node[left]{$Y$} -- (mux.rpin 1) coordinate (Y)

        (d0) node[above]{1}
        (d1) node[above]{$\overline{Z}$}
        (d2) node[above]{$\overline{Z}$}
        (d3) node[above]{1}
        (d4) node[above]{Z}
        (d5) node[above]{0}
        (d6) node[above]{0}
        (d7) node[above]{Z}

        (S2) node[below]{W}
        (S1) node[below]{X}
        (S0) node[below]{Y}

        (Y) node[right]{$f$}

        (d0) -- ++(-0.5,0) node[vcc]{$V_{cc}$}
        (d1) ++(-0.5,0) node[jump crossing](xing1){}
        (d2) ++(-0.5,0) node[jump crossing](xing2){}
        (d3) -- ++(-0.5,0) -- (xing2.south)
        (xing2.north) -- (xing1.south)
        (d0) ++(-0.5,0) -- (xing1.north)
        (xing1.east) -- (d1)
        (xing2.east) -- (d2)
        (d6) -- ++(-0.5,0) -- ++(0,-0.2) node[tlground](gnd){}
        (d5) -- ++(-0.5,0) -- (d6-|gnd)
        (d7) -- ++(-1,0) coordinate(Z) -- (Z|-d4) node[not port, scale=0.75, anchor=in, rotate=90](not){} -- (d4)
        (not.out) -- (not.out|-d2) -- (xing2.west)
        (not.out|-d2) -- (not.out|-d1) -- (xing1.west)
        (Z) -- ++(0,-1) node[below]{Z}
        ;
    \end{circuitikz}
\end{center}

\pagebreak

\section{Using two 4:1 muxes, generate the $B_O$ (Borrow-Out) and $C_O$ (Carry-Out) outputs for a 1-bit full subtractor and 1-bit full adder respectively where X is the minuend, Y is the subtrahend or Y and Y are the addends and bc is the borrow-in or carry-in.}
$$B_O=\overline{X}\cdot b_c+\overline{X}\cdot Y+Y\cdot b_c$$
$$C_O=X\cdot Y+X\cdot b_c+Y\cdot b_c$$

\begin{table}[H]
    \centering
    \begin{tabular}{cc:c|cc}
        \toprule
        X&Y&$b_c$&$B_O$&$C_O$\\
        \midrule
        0&0&0&0&0\\
        0&0&1&1&0\\
        \greyrule
        0&1&0&1&0\\
        0&1&1&1&1\\
        \bottomrule
    \end{tabular}
    \quad
    \begin{tabular}{cc:c|cc}
        \toprule
        X&Y&$b_c$&$B_O$&$C_O$\\
        \midrule
        1&0&0&0&0\\
        1&0&1&0&1\\
        \greyrule
        1&1&0&0&1\\
        1&1&1&1&1\\
        \bottomrule
    \end{tabular}
\end{table}

\,\\
\begin{center}
    \begin{circuitikz}
        \tikzset{mux/.style={muxdemux,muxdemux def={Lh= 6,Rh= 4,NL=6,NB=2,w=2},no input leads}}
        \draw
        (0,0) node[mux](mux){}

        (mux.blpin 2) node[right]{$d_0$} -- (mux.lpin 2) coordinate (1d0)
        (mux.blpin 3) node[right]{$d_1$} -- (mux.lpin 3) coordinate (1d1)
        (mux.blpin 4) node[right]{$d_2$} -- (mux.lpin 4) coordinate (1d2)
        (mux.blpin 5) node[right]{$d_3$} -- (mux.lpin 5) coordinate (1d3)

        (mux.bbpin 1) node[above]{\small{$S_1$}} -- (mux.bpin 1) coordinate (1S1)
        (mux.bbpin 2) node[above]{\small{$S_0$}} -- (mux.bpin 2) coordinate (1S0)

        (mux.brpin 1) node[left]{$Y$} -- (mux.rpin 1) coordinate (1Y)

        (4,0) node[mux](mux2){}

        (mux2.blpin 2) node[right]{$d_0$} -- (mux2.lpin 2) coordinate (2d0)
        (mux2.blpin 3) node[right]{$d_1$} -- (mux2.lpin 3) coordinate (2d1)
        (mux2.blpin 4) node[right]{$d_2$} -- (mux2.lpin 4) coordinate (2d2)
        (mux2.blpin 5) node[right]{$d_3$} -- (mux2.lpin 5) coordinate (2d3)

        (mux2.bbpin 1) node[above]{\small{$S_1$}} -- (mux2.bpin 1) coordinate (2S1)
        (mux2.bbpin 2) node[above]{\small{$S_0$}} -- (mux2.bpin 2) coordinate (2S0)

        (mux2.brpin 1) node[left]{$Y$} -- (mux2.rpin 1) coordinate (2Y)

        (1S1-|1S0) node[jump crossing](xing){}
        (xing.north)-- (1S0)
        (xing.east) -- (2S1)
        (xing.west) -- (1S1)
        (xing.south) -- ++(0,-0.25) node[circ](cntk){}
        (2S0) -- (2S0|-cntk) -- (mux.lpin 1|-cntk) node[left]{Y}
        (1S1) node[circ]{} -- (mux.lpin 1|-2S1) node[left]{X}

        (1Y) node[right]{$B_O$}
        (2Y) node[right]{$C_O$}

%        (1d0) node[above]{$b_c$}
        (1d1) node[left]{1}
        (1d2) node[above]{0}
%        (1d3) node[above]{$b_c$}
        
        (2d0) node[above]{0}
        (2d1) node[above]{$b_c$}
%        (2d2) node[above]{$b_c$}
        (2d3) node[above]{1}
        
        (1d1) node[vcc]{}

        (2d0) -- ++(-0.5,0) node[ground]{}
        (2d1) -- (2d2)
        (2d3) -- ++(-0.5,0) node[vcc]{}

        (1d0) -- ++(-0.5,0) node[above](bc){$b_c$} -- (bc|-1d3) -- (1d3)
        (1d2) -- ++(0,-0.2) node[tlground]{}
        ;
    \end{circuitikz}
\end{center}

\pagebreak

\section{Using an 8:1 mux, create a circuit to generate the segment $g$ outputs for a 7-segment display where $g=\overline{B}\cdot \overline{D}+C\cdot \overline{D}$.}
\begin{table}[H]
    \centering
    \begin{tabular}{ccc:c|c}
        \toprule
        A&B&C&D&$g$\\
        \midrule
        0&0&0&0&1\\
        0&0&0&1&0\\
        \greyrule
        0&0&1&0&1\\
        0&0&1&1&0\\
        \greyrule
        0&1&0&0&0\\
        0&1&0&1&0\\
        \greyrule
        0&1&1&0&1\\
        0&1&1&1&0\\
        \bottomrule
    \end{tabular}
    \quad
    \begin{tabular}{ccc:c|c}
        \toprule
        A&B&C&D&$g$\\
        \midrule
        1&0&0&0&1\\
        1&0&0&1&0\\
        \greyrule
        1&0&1&0&1\\
        1&0&1&1&0\\
        \greyrule
        1&1&0&1&0\\
        1&1&0&0&0\\
        \greyrule
        1&1&1&0&1\\
        1&1&1&1&0\\
        \bottomrule
    \end{tabular}
\end{table}
\,\\
\begin{center}
    \begin{circuitikz}
        \tikzset{mux/.style={muxdemux,muxdemux def={Lh= 10,Rh= 7,NL=10},no input leads}}
        \draw
        (0,0) node[mux](mux){}

        (mux.blpin 2) node[right]{$d_0$} -- (mux.lpin 2) coordinate (d0)
        (mux.blpin 3) node[right]{$d_1$} -- (mux.lpin 3) coordinate (d1)
        (mux.blpin 4) node[right]{$d_2$} -- (mux.lpin 4) coordinate (d2)
        (mux.blpin 5) node[right]{$d_3$} -- (mux.lpin 5) coordinate (d3)
        (mux.blpin 6) node[right]{$d_4$} -- (mux.lpin 6) coordinate (d4)
        (mux.blpin 7) node[right]{$d_5$} -- (mux.lpin 7) coordinate (d5)
        (mux.blpin 8) node[right]{$d_6$} -- (mux.lpin 8) coordinate (d6)
        (mux.blpin 9) node[right]{$d_7$} -- (mux.lpin 9) coordinate (d7)

        (mux.bbpin 1) node[above]{\small{$S_2$}} -- (mux.bpin 1) coordinate (S2)
        (mux.bbpin 2) node[above]{\small{$S_1$}} -- (mux.bpin 2) coordinate (S1)
        (mux.bbpin 3) node[above]{\small{$S_0$}} -- (mux.bpin 3) coordinate (S0)

        (mux.brpin 1) node[left]{$Y$} -- (mux.rpin 1) coordinate (Y)

        (d0) node[above]{$\overline{D}$}
        (d1) node[above]{$\overline{D}$}
        (d2) node[above]{0}
        (d3) node[above]{$\overline{D}$}
        (d4) node[above]{$\overline{D}$}
        (d5) node[above]{$\overline{D}$}
        (d6) node[above]{0}
        (d7) node[above]{$\overline{D}$}

        (S2) node[below]{A}
        (S1) node[below]{B}
        (S0) node[below]{C}

        (Y) node[right]{$g$}

        (d7) -- ++(-1,0) node[not port,anchor=out, rotate=90](not){}
        (not.in) node[below]{D}
        (d6) -- ++(-0.5,0) -- ++(0,-0.2) node[tlground](gnd){}
        (d2) -- ++(-0.5,0) -- ++(0,-0.2) node[tlground](gnd){}
        (d0) -- ++(-1,0) -- (not.out)
        (d1) -- ++(-1,0)
        (d3) -- ++(-1,0)
        (d4) -- ++(-1,0)
        (d5) -- ++(-1,0)
        ;
    \end{circuitikz}
\end{center}

\pagebreak

\section{Using an 8:1 mux, create a circuit to generate}
$$f(ABCD)=A\cdot D+\overline{A}\cdot\overline{B}\cdot \overline{D}+\overline{B}\cdot C+\overline{A}\cdot C\cdot\overline{D}$$

\begin{table}[H]
    \centering
    \begin{tabular}{ccc:c|c}
        \toprule
        A&B&C&D&$f$\\
        \midrule
        0&0&0&0&1\\
        0&0&0&1&0\\
        \greyrule
        0&0&1&0&1\\
        0&0&1&1&1\\
        \greyrule
        0&1&0&0&0\\
        0&1&0&1&0\\
        \greyrule
        0&1&1&0&1\\
        0&1&1&1&0\\
        \bottomrule
    \end{tabular}
    \quad
    \begin{tabular}{ccc:c|c}
        \toprule
        A&B&C&D&$f$\\
        \midrule
        1&0&0&0&0\\
        1&0&0&1&1\\
        \greyrule
        1&0&1&0&1\\
        1&0&1&1&1\\
        \greyrule
        1&1&0&0&0\\
        1&1&0&1&1\\
        \greyrule
        1&1&1&0&0\\
        1&1&1&1&1\\
        \bottomrule
    \end{tabular}
\end{table}
\,\\
\begin{center}
    \begin{circuitikz}
        \tikzset{mux/.style={muxdemux,muxdemux def={Lh= 10,Rh= 7,NL=10},no input leads}}
        \draw
        (0,0) node[mux](mux){}

        (mux.blpin 2) node[right]{$d_0$} -- (mux.lpin 2) coordinate (d0)
        (mux.blpin 3) node[right]{$d_1$} -- (mux.lpin 3) coordinate (d1)
        (mux.blpin 4) node[right]{$d_2$} -- (mux.lpin 4) coordinate (d2)
        (mux.blpin 5) node[right]{$d_3$} -- (mux.lpin 5) coordinate (d3)
        (mux.blpin 6) node[right]{$d_4$} -- (mux.lpin 6) coordinate (d4)
        (mux.blpin 7) node[right]{$d_5$} -- (mux.lpin 7) coordinate (d5)
        (mux.blpin 8) node[right]{$d_6$} -- (mux.lpin 8) coordinate (d6)
        (mux.blpin 9) node[right]{$d_7$} -- (mux.lpin 9) coordinate (d7)

        (mux.bbpin 1) node[above]{\small{$S_2$}} -- (mux.bpin 1) coordinate (S2)
        (mux.bbpin 2) node[above]{\small{$S_1$}} -- (mux.bpin 2) coordinate (S1)
        (mux.bbpin 3) node[above]{\small{$S_0$}} -- (mux.bpin 3) coordinate (S0)

        (mux.brpin 1) node[left]{$Y$} -- (mux.rpin 1) coordinate (Y)

        (d0) node[above]{$\overline{D}$}
        (d1) node[above]{1}
        (d2) node[above]{0}
        (d3) node[above]{$\overline{D}$}
        (d4) node[above]{D}
        (d5) node[above]{1}
        (d6) node[above]{D}
        (d7) node[above]{D}

        (S2) node[below]{A}
        (S1) node[below]{B}
        (S0) node[below]{C}

        (Y) node[right]{$f$}

        (d0) -- ++(-1.5,0)
        (d6) -- ++(-1.5,0)
        (d3) ++(-1,0) node[jump crossing](xing3){}
        (d4) ++(-1,0) node[jump crossing](xing4){}
        (xing3.north) -- (xing3|-d1) node[vcc, label=45:$V_{cc}$]{} -- (d1)
        (xing4.south) -- (xing4|-d5) -- (d5)
        (xing3.east) -- (d3)
        (xing4.east) -- (d4)
        (xing3.south) -- (xing4.north)
        (d2) -- ++(-0.5,0) -- ++(0,-0.2) node[tlground](gnd){}
        (d7) -- ++(-1.5,0) coordinate(D) -- (D|-d4) node[not port, scale=0.4, anchor=in, rotate=90](not){} -- (xing4.west)
        (D) -- ++(0,-1) node[below]{D}
        (xing3.west) -- (not.out|-d3) -- (not.out)
        (not.out|-d3) -- (not.out|-d0)
        ;
    \end{circuitikz}
\end{center}

\pagebreak

\section{Using an 8:1 mux, create a circuit to generate the segment b outputs for a 7-segment display.}
Given ABCD = 0, 1, 2, \dots, F,  the segment $b$ outputs are $(F9C8)_{16}$ \big(e.g. the MSB of $(F)_{16} =$ output for ABCD $= (0000)_2$ and the LSB of $(8)_{16} =$ output for ABCD $= (1111)_2$\big).
$$(F9C8)_{16}=(\groupify{4}{1111 1001 1100 1000})_2$$


\begin{table}[H]
    \centering
    \begin{tabular}{ccc:c|c}
        \toprule
        A&B&C&D&$b$\\
        \midrule
        0&0&0&0&1\\
        0&0&0&1&1\\
        \greyrule
        0&0&1&0&1\\
        0&0&1&1&1\\
        \greyrule
        0&1&0&0&1\\
        0&1&0&1&0\\
        \greyrule
        0&1&1&0&0\\
        0&1&1&1&1\\
        \bottomrule
    \end{tabular}
    \quad
    \begin{tabular}{ccc:c|c}
        \toprule
        A&B&C&D&$b$\\
        \midrule
        1&0&0&0&1\\
        1&0&0&1&1\\
        \greyrule
        1&0&1&0&0\\
        1&0&1&1&0\\
        \greyrule
        1&1&0&0&1\\
        1&1&0&1&0\\
        \greyrule
        1&1&1&0&0\\
        1&1&1&1&0\\
        \bottomrule
    \end{tabular}
\end{table}
\,\\
\begin{center}
    \begin{circuitikz}
        \tikzset{mux/.style={muxdemux,muxdemux def={Lh= 10,Rh= 7,NL=10},no input leads}}
        \draw
        (0,0) node[mux](mux){}

        (mux.blpin 2) node[right]{$d_0$} -- (mux.lpin 2) coordinate (d0)
        (mux.blpin 3) node[right]{$d_1$} -- (mux.lpin 3) coordinate (d1)
        (mux.blpin 4) node[right]{$d_2$} -- (mux.lpin 4) coordinate (d2)
        (mux.blpin 5) node[right]{$d_3$} -- (mux.lpin 5) coordinate (d3)
        (mux.blpin 6) node[right]{$d_4$} -- (mux.lpin 6) coordinate (d4)
        (mux.blpin 7) node[right]{$d_5$} -- (mux.lpin 7) coordinate (d5)
        (mux.blpin 8) node[right]{$d_6$} -- (mux.lpin 8) coordinate (d6)
        (mux.blpin 9) node[right]{$d_7$} -- (mux.lpin 9) coordinate (d7)

        (mux.bbpin 1) node[above]{\small{$S_2$}} -- (mux.bpin 1) coordinate (S2)
        (mux.bbpin 2) node[above]{\small{$S_1$}} -- (mux.bpin 2) coordinate (S1)
        (mux.bbpin 3) node[above]{\small{$S_0$}} -- (mux.bpin 3) coordinate (S0)

        (mux.brpin 1) node[left]{$Y$} -- (mux.rpin 1) coordinate (Y)

        (d0) node[above]{1}
        (d1) node[above]{1}
        (d2) node[above]{$\overline{D}$}
        (d3) node[above]{D}
        (d4) node[above]{1}
        (d5) node[above]{0}
        (d6) node[above]{$\overline{D}$}
        (d7) node[above]{0}

        (S2) node[below]{A}
        (S1) node[below]{B}
        (S0) node[below]{C}

        (Y) node[right]{$b$}

        (d2) -- ++(-0.5,0) node[jump crossing, anchor=east](xing2){}
        (d3) -- ++(-0.5,0) node[jump crossing, anchor=east](xing3){}
        (d6) -- ++(-0.5,0) node[jump crossing, anchor=east](xing6){}
        (d4) -- (d4-|xing3.south) -- (xing3.south)
        (xing2.south) -- (xing3.north)
        (xing2.north) -- (xing2.north|-d0) node[vcc](vcc){$V_{cc}$}
        (d0) -- (d0-|vcc)
        (d1) -- (d1-|vcc)
        (xing6.north) -- (xing6.north|-d5) -- (d5)
        (d7) -- (d7-|xing6.south) node[ground]{} -- (xing6.south)
        (xing3.west) ++(-1.5,0) coordinate(Dup) -- (Dup|-S2) node[below]{D}
        (Dup|-S2) ++(0,0.25) -- ++(0.75,0) node[not port, anchor=in, scale=0.9, rotate=90](not){}
        (not.out) -- (not.out|-xing6.west) -- (xing6.west)
        (d3-|not.out) node[jump crossing](xing3l){}
        (xing3l.east) -- (xing3.west)
        (xing3l.west) -- (Dup)
        (xing3l.north) -- (xing3l.north|-xing2.west) -- (xing2.west)
        (xing3l.south) -- (not.out)
        
        ;
    \end{circuitikz}
\end{center}

\pagebreak

\section{Given inputs A, B, and C where A is the MSB, use a 3:8 demux to generate $X=\sum(m_0,m_1,m_2,m_4,m_6)$}

\,\\

\begin{center}
    \begin{circuitikz}
        \tikzset{demux/.style={muxdemux, muxdemux def={Lh=6, Rh=8, NL=5, NB=0, NR=10},no input leads}}
        \ctikzset{tripoles/american or port/height=1.6}
        \draw
        (0,0) node[demux](demux){}

        (demux.brpin 2) node[left]{$Y_0$} -- (demux.rpin 2) coordinate (Y0)
        (demux.brpin 3) node[left]{$Y_1$} -- (demux.rpin 3) coordinate (Y1)
        (demux.brpin 4) node[left]{$Y_2$} -- (demux.rpin 4) coordinate (Y2)
        (demux.brpin 5) node[left]{$Y_3$} -- (demux.rpin 5) coordinate (Y3)
        (demux.brpin 6) node[left]{$Y_4$} -- (demux.rpin 6) coordinate (Y4)
        (demux.brpin 7) node[left]{$Y_5$} -- (demux.rpin 7) coordinate (Y5)
        (demux.brpin 8) node[left]{$Y_6$} -- (demux.rpin 8) coordinate (Y6)
        (demux.brpin 9) node[left]{$Y_7$} -- (demux.rpin 9) coordinate (Y7)
        (demux.blpin 2) node[right]{$S_2$} -- (demux.lpin 2) coordinate (S2)
        (demux.blpin 3) node[right]{$S_1$} -- (demux.lpin 3) coordinate (S1)
        (demux.blpin 4) node[right]{$S_0$} -- (demux.lpin 4) coordinate (S0)

        (demux.brpin 2) node[ocirc,right]{}
        (demux.brpin 3) node[ocirc,right]{}
        (demux.brpin 4) node[ocirc,right]{}
        (demux.brpin 5) node[ocirc,right]{}
        (demux.brpin 6) node[ocirc,right]{}
        (demux.brpin 7) node[ocirc,right]{}
        (demux.brpin 8) node[ocirc,right]{}
        (demux.brpin 9) node[ocirc,right]{}

        (S2) node[left]{A}
        (S1) node[left]{B}
        (S0) node[left]{C}

        (Y0) ++(0.5,0.5) node[or port, number inputs=5, anchor=in 1, rotate=90](nand1){}
        (nand1.out) node[above]{X}
        (nand1.bin 1) node[ocirc,below]{}
        (nand1.bin 2) node[ocirc,below]{}
        (nand1.bin 3) node[ocirc,below]{}
        (nand1.bin 4) node[ocirc,below]{}
        (nand1.bin 5) node[ocirc,below]{}
        
        (nand1.in 1) -- (nand1.bin 1|-Y0) -- (Y0)
        (nand1.in 2) -- (nand1.bin 2|-Y1) -- (Y1)
        (nand1.in 3) -- (nand1.bin 3|-Y2) -- (Y2)
        (nand1.in 4) -- (nand1.bin 4|-Y4) -- (Y4)
        (nand1.in 5) -- (nand1.bin 5|-Y6) -- (Y6)
        ;



    \end{circuitikz}
\end{center}

\pagebreak

\section{Using a 3:8 demux, generate a complemented Gray code}

\begin{table}[H]
    \centering
    \begin{tabular}{ccc:c|ccc:c}
        \toprule
        A&B&C&D&W&X&Y&Z\\
        \midrule
        0&0&0&0&1&1&1&1\\
        0&0&0&1&1&1&1&0\\
        \greyrule
        0&0&1&0&1&1&0&0\\
        0&0&1&1&1&1&0&1\\
        \greyrule
        0&1&0&0&1&0&0&1\\
        0&1&0&1&1&0&0&0\\
        \greyrule
        0&1&1&0&1&0&1&0\\
        0&1&1&1&1&0&1&1\\
        \bottomrule
    \end{tabular}
    \quad
    \begin{tabular}{ccc:c|ccc:c}
        \toprule
        A&B&C&D&W&X&Y&Z\\
        \midrule
        1&0&0&0&0&0&1&1\\
        1&0&0&1&0&0&1&0\\
        \greyrule
        1&0&1&0&0&0&0&0\\
        1&0&1&1&0&0&0&1\\
        \greyrule
        1&1&0&0&0&1&0&1\\
        1&1&0&1&0&1&0&0\\
        \greyrule
        1&1&1&0&0&1&1&0\\
        1&1&1&1&0&1&1&1\\
        \bottomrule
    \end{tabular}
\end{table}

Observing the truth table:
$$Z=C\otimes  D$$

\begin{center}
    \begin{circuitikz}
        \tikzset{demux/.style={muxdemux, muxdemux def={Lh=6, Rh=8, NL=5, NB=0, NR=10},no input leads}}
        \ctikzset{tripoles/american or port/height=1.25}
        \draw
        (0,0) node[demux](demux){}

        (demux.brpin 2) node[left]{$Y_0$} -- (demux.rpin 2) coordinate (Y0)
        (demux.brpin 3) node[left]{$Y_1$} -- (demux.rpin 3) coordinate (Y1)
        (demux.brpin 4) node[left]{$Y_2$} -- (demux.rpin 4) coordinate (Y2)
        (demux.brpin 5) node[left]{$Y_3$} -- (demux.rpin 5) coordinate (Y3)
        (demux.brpin 6) node[left]{$Y_4$} -- (demux.rpin 6) coordinate (Y4)
        (demux.brpin 7) node[left]{$Y_5$} -- (demux.rpin 7) coordinate (Y5)
        (demux.brpin 8) node[left]{$Y_6$} -- (demux.rpin 8) coordinate (Y6)
        (demux.brpin 9) node[left]{$Y_7$} -- (demux.rpin 9) coordinate (Y7)
        (demux.blpin 2) node[right]{$S_2$} -- (demux.lpin 2) coordinate (S2)
        (demux.blpin 3) node[right]{$S_1$} -- (demux.lpin 3) coordinate (S1)
        (demux.blpin 4) node[right]{$S_0$} -- (demux.lpin 4) coordinate (S0)

        (demux.brpin 2) node[ocirc,right]{}
        (demux.brpin 3) node[ocirc,right]{}
        (demux.brpin 4) node[ocirc,right]{}
        (demux.brpin 5) node[ocirc,right]{}
        (demux.brpin 6) node[ocirc,right]{}
        (demux.brpin 7) node[ocirc,right]{}
        (demux.brpin 8) node[ocirc,right]{}
        (demux.brpin 9) node[ocirc,right]{}

        (S2) coordinate(A) node[above]{A}
        (S1) coordinate(B) node[above]{B}
        (S0) coordinate(C) node[above]{C}
        (C) ++(0,-0.65) coordinate(D) node[above]{D}

        (Y0) ++(0.5,0.5) node[or port, number inputs=4, anchor=in 1, rotate=90](nand1){}
        (nand1.out) node[above]{W}
        (nand1.bin 1) node[ocirc,below]{}
        (nand1.bin 2) node[ocirc,below]{}
        (nand1.bin 3) node[ocirc,below]{}
        (nand1.bin 4) node[ocirc,below]{}
        
        (nand1.in 1) -- (nand1.bin 1|-Y0) node[circ](XY0){} -- (Y0)
        (nand1.in 2) -- (nand1.bin 2|-Y1) node[circ](XY1){} -- (Y1)
        (nand1.in 3) -- (nand1.bin 3|-Y2) node[circ](XY2){} -- (Y2)
        (nand1.in 4) -- (nand1.bin 4|-Y3) node[circ](XY3){} -- (Y3)
        
        (nand1.in 4) ++(0.45,0) node[or port, number inputs=4, anchor=in 1, rotate=90](nand2){}
        (nand2.out) node[above]{X}
        (nand2.bin 1) node[ocirc,below]{}
        (nand2.bin 2) node[ocirc,below]{}
        (nand2.bin 3) node[ocirc,below]{}
        (nand2.bin 4) node[ocirc,below]{}

        (nand2.in 1) -- (nand2.in 1|-Y0) node[circ](YY0){} -- (XY0)
        (nand2.in 2) -- (nand2.in 2|-Y1) node[circ](YY1){} -- (XY1)
        (nand2.in 3) -- (nand2.in 3|-Y6) node[circ](YY6){} -- (Y6)
        (nand2.in 4) -- (nand2.in 4|-Y7) node[circ](YY7){} -- (Y7)

        (nand2.in 4) ++(0.45,0) node[or port, number inputs=4, anchor=in 1, rotate=90](nand3){}
        (nand3.out) node[above]{Y}
        (nand3.bin 1) node[ocirc,below]{}
        (nand3.bin 2) node[ocirc,below]{}
        (nand3.bin 3) node[ocirc,below]{}
        (nand3.bin 4) node[ocirc,below]{}
        (nand3.in 1) -- (nand3.in 1|-Y0) node[circ]{} -- (YY0)
        (nand3.in 2) -- (nand3.in 2|-Y3) node[circ]{} -- (XY3)
        (nand3.in 3) -- (nand3.in 3|-Y4) node[circ]{} -- (Y4)
        (nand3.in 4) -- (nand3.in 4|-Y7) node[circ]{} -- (YY7)
        
        (C) node[xor port, rotate=180, anchor=in 2,scale=1.2](xor){}
        (xor.bout) node[ocirc,left]{}
        (xor.out) node[left]{Z}
        ;
    \end{circuitikz}
\end{center}

\pagebreak


\section{Using a 3:8 demux, generate these functions of A, B and C:}
$$f=\sum(m_0,m_1,m_3,m_7)$$
$$g=\sum(m_1,m_2,m_5,m_6)$$
$$h=\sum(m_0,m_2,m_4,m_5,m_6)$$

\,\\
\begin{center}
    \begin{circuitikz}
        \tikzset{demux/.style={muxdemux, muxdemux def={Lh=6, Rh=8, NL=5, NB=0, NR=10},no input leads}}
        \ctikzset{tripoles/american or port/height=1.25}
        \draw
        (0,0) node[demux](demux){}

        (demux.brpin 2) node[left]{$Y_0$} -- (demux.rpin 2) coordinate (Y0)
        (demux.brpin 3) node[left]{$Y_1$} -- (demux.rpin 3) coordinate (Y1)
        (demux.brpin 4) node[left]{$Y_2$} -- (demux.rpin 4) coordinate (Y2)
        (demux.brpin 5) node[left]{$Y_3$} -- (demux.rpin 5) coordinate (Y3)
        (demux.brpin 6) node[left]{$Y_4$} -- (demux.rpin 6) coordinate (Y4)
        (demux.brpin 7) node[left]{$Y_5$} -- (demux.rpin 7) coordinate (Y5)
        (demux.brpin 8) node[left]{$Y_6$} -- (demux.rpin 8) coordinate (Y6)
        (demux.brpin 9) node[left]{$Y_7$} -- (demux.rpin 9) coordinate (Y7)
        (demux.blpin 2) node[right]{$S_2$} -- (demux.lpin 2) coordinate (S2)
        (demux.blpin 3) node[right]{$S_1$} -- (demux.lpin 3) coordinate (S1)
        (demux.blpin 4) node[right]{$S_0$} -- (demux.lpin 4) coordinate (S0)

        (demux.brpin 2) node[ocirc,right]{}
        (demux.brpin 3) node[ocirc,right]{}
        (demux.brpin 4) node[ocirc,right]{}
        (demux.brpin 5) node[ocirc,right]{}
        (demux.brpin 6) node[ocirc,right]{}
        (demux.brpin 7) node[ocirc,right]{}
        (demux.brpin 8) node[ocirc,right]{}
        (demux.brpin 9) node[ocirc,right]{}

        (S2) node[left]{A}
        (S1) node[left]{B}
        (S0) node[left]{C}

        (Y0) ++(0.5,0.5) node[or port, number inputs=4, anchor=in 1, rotate=90](nand1){}
        (nand1.out) node[above]{$f$}
        (nand1.bin 1) node[ocirc,below]{}
        (nand1.bin 2) node[ocirc,below]{}
        (nand1.bin 3) node[ocirc,below]{}
        (nand1.bin 4) node[ocirc,below]{}
        
        (nand1.in 1) -- (nand1.bin 1|-Y0) node[circ](fY0){} -- (Y0)
        (nand1.in 2) -- (nand1.bin 2|-Y1) node[circ](fY1){} -- (Y1)
        (nand1.in 3) -- (nand1.bin 3|-Y3) node[circ](fY3){} -- (Y3)
        (nand1.in 4) -- (nand1.bin 4|-Y7) node[circ](fY7){} -- (Y7)
        
        (nand1.in 4) ++(0.45,0) node[or port, number inputs=4, anchor=in 1, rotate=90](nand2){}
        (nand2.out) node[above]{$g$}
        (nand2.bin 1) node[ocirc,below]{}
        (nand2.bin 2) node[ocirc,below]{}
        (nand2.bin 3) node[ocirc,below]{}
        (nand2.bin 4) node[ocirc,below]{}

        (nand2.in 1) -- (nand2.in 1|-Y1) node[circ](gY1){} -- (fY1)
        (nand2.in 2) -- (nand2.in 2|-Y2) node[circ](gY2){} -- (Y2)
        (nand2.in 3) -- (nand2.in 3|-Y5) node[circ](gY5){} -- (Y5)
        (nand2.in 4) -- (nand2.in 4|-Y6) node[circ](gY6){} -- (Y6)
        ;
        \ctikzset{tripoles/american or port/height=1.5}
        \draw
        (nand2.in 4) ++(0.45,0) node[or port, number inputs=5, anchor=in 1, rotate=90](nand3){}
        (nand3.out) node[above]{$h$}
        (nand3.bin 1) node[ocirc,below]{}
        (nand3.bin 2) node[ocirc,below]{}
        (nand3.bin 3) node[ocirc,below]{}
        (nand3.bin 4) node[ocirc,below]{}
        (nand3.bin 5) node[ocirc,below]{}
        (nand3.in 1) -- (nand3.in 1|-fY0) node[circ]{} -- (fY0)
        (nand3.in 2) -- (nand3.in 2|-gY2) node[circ]{} -- (gY2)
        (nand3.in 3) -- (nand3.in 3|-Y4) node[circ]{} -- (Y4)
        (nand3.in 4) -- (nand3.in 4|-gY5) node[circ]{} -- (gY5)
        (nand3.in 5) -- (nand3.in 5|-gY6) node[circ]{} -- (gY6)
        
        ;
    \end{circuitikz}
\end{center}

\pagebreak

\section{Using a 3:8 demux generate the following truth table:}
\begin{table}[H]
    \centering
    \begin{tabular}{ccc|c:c}
        \toprule
        A&B&C&$X$&$e$\\
        \midrule
        0&0&0&1&1\\
        0&0&1&0&0\\
        0&1&0&0&1\\
        0&1&1&0&1\\
        \bottomrule
    \end{tabular}
    \quad
    \begin{tabular}{ccc|c:c}
        \toprule
        A&B&C&$X$&$e$\\
        \midrule
        1&0&0&1&1\\
        1&0&1&1&1\\
        1&1&0&0&0\\
        1&1&1&1&1\\
        \bottomrule
    \end{tabular}
\end{table}

\,\\
\begin{center}
    \begin{circuitikz}
        
        \tikzset{demux/.style={muxdemux, muxdemux def={Lh=6, Rh=8, NL=5, NB=0, NR=10},no input leads}}
        \ctikzset{tripoles/american or port/height=1.25}

        \draw
        (0,0) node[demux](demux){}

        (demux.brpin 2) node[left]{$Y_0$} -- (demux.rpin 2) coordinate (Y0)
        (demux.brpin 3) node[left]{$Y_1$} -- (demux.rpin 3) coordinate (Y1)
        (demux.brpin 4) node[left]{$Y_2$} -- (demux.rpin 4) coordinate (Y2)
        (demux.brpin 5) node[left]{$Y_3$} -- (demux.rpin 5) coordinate (Y3)
        (demux.brpin 6) node[left]{$Y_4$} -- (demux.rpin 6) coordinate (Y4)
        (demux.brpin 7) node[left]{$Y_5$} -- (demux.rpin 7) coordinate (Y5)
        (demux.brpin 8) node[left]{$Y_6$} -- (demux.rpin 8) coordinate (Y6)
        (demux.brpin 9) node[left]{$Y_7$} -- (demux.rpin 9) coordinate (Y7)
        (demux.blpin 2) node[right]{$S_2$} -- (demux.lpin 2) coordinate (S2)
        (demux.blpin 3) node[right]{$S_1$} -- (demux.lpin 3) coordinate (S1)
        (demux.blpin 4) node[right]{$S_0$} -- (demux.lpin 4) coordinate (S0)

        (demux.brpin 2) node[ocirc,right]{}
        (demux.brpin 3) node[ocirc,right]{}
        (demux.brpin 4) node[ocirc,right]{}
        (demux.brpin 5) node[ocirc,right]{}
        (demux.brpin 6) node[ocirc,right]{}
        (demux.brpin 7) node[ocirc,right]{}
        (demux.brpin 8) node[ocirc,right]{}
        (demux.brpin 9) node[ocirc,right]{}

        (S2) node[left]{A}
        (S1) node[left]{B}
        (S0) node[left]{C}

        (Y0) ++(0.5,0.5) node[or port, number inputs=4, anchor=in 1, rotate=90](nand1){}
        (nand1.out) node[above]{$X$}
        (nand1.bin 1) node[ocirc,below]{}
        (nand1.bin 2) node[ocirc,below]{}
        (nand1.bin 3) node[ocirc,below]{}
        (nand1.bin 4) node[ocirc,below]{}
        
        (nand1.in 1) -- (nand1.bin 1|-Y0) node[circ](XY0){} -- (Y0)
        (nand1.in 2) -- (nand1.bin 2|-Y4) node[circ](XY4){} -- (Y4)
        (nand1.in 3) -- (nand1.bin 3|-Y5) node[circ](XY5){} -- (Y5)
        (nand1.in 4) -- (nand1.bin 4|-Y7) node[circ](XY7){} -- (Y7)
        ;

        \ctikzset{tripoles/american or port/height=1.85}

        \draw
        (nand1.in 4) ++(0.45,0) node[or port, number inputs=6, anchor=in 1, rotate=90](nand2){}
        (nand2.out) node[above]{$e$}
        (nand2.bin 1) node[ocirc,below]{}
        (nand2.bin 2) node[ocirc,below]{}
        (nand2.bin 3) node[ocirc,below]{}
        (nand2.bin 4) node[ocirc,below]{}
        (nand2.bin 5) node[ocirc,below]{}
        (nand2.bin 6) node[ocirc,below]{}

        (nand2.in 1) -- (nand2.in 1|-Y0) node[circ]{} -- (XY0)
        (nand2.in 2) -- (nand2.in 2|-Y2) node[circ]{} -- (Y2)
        (nand2.in 3) -- (nand2.in 3|-Y3) node[circ]{} -- (Y3)
        (nand2.in 4) -- (nand2.in 4|-Y4) node[circ]{} -- (XY4)
        (nand2.in 5) -- (nand2.in 5|-Y5) node[circ]{} -- (XY5)
        (nand2.in 6) -- (nand2.in 6|-Y7) node[circ]{} -- (XY7)
        ;
    \end{circuitikz}
\end{center}

\pagebreak

\setcounter{section}{10}
\subsection{Using a 3:8 demux, transpose input ABCD then add two, modulo 16 to created output WXYZ.\\(i.e. WXYZ = (DCBA + 2)mod 16).}

\begin{table}[H]
    \centering
    \begin{tabular}{ccc:c|c:ccc}
        \toprule
        A&B&C&D&W&X&Y&Z\\
        \midrule
        0&0&0&0&0&0&1&0\\
        0&0&0&1&1&0&1&0\\
        \greyrule
        0&0&1&0&0&1&1&0\\
        0&0&1&1&1&1&1&0\\
        \greyrule
        0&1&0&0&0&1&0&0\\
        0&1&0&1&1&1&0&0\\
        \greyrule
        0&1&1&0&1&0&0&0\\
        0&1&1&1&0&0&0&0\\
        \bottomrule
    \end{tabular}
    \quad
    \begin{tabular}{ccc:c|c:ccc}
        \toprule
        A&B&C&D&W&X&Y&Z\\
        \midrule
        1&0&0&0&0&0&1&1\\
        1&0&0&1&1&0&1&1\\
        \greyrule
        1&0&1&0&0&1&1&1\\
        1&0&1&1&1&1&1&1\\
        \greyrule
        1&1&0&0&0&1&0&1\\
        1&1&0&1&1&1&0&1\\
        \greyrule
        1&1&1&0&1&0&0&1\\
        1&1&1&1&0&0&0&1\\
        \bottomrule
    \end{tabular}
\end{table}

Observing the truth table:
$$W=(B\cdot C)\oplus D$$
\begin{center}
    \begin{circuitikz}
        \tikzset{demux/.style={muxdemux, muxdemux def={Lh=6, Rh=8, NL=5, NB=0, NR=10},no input leads}}
        \ctikzset{tripoles/american or port/height=1.25}
        \draw
        (0,0) node[demux](demux){}

        (demux.brpin 2) node[left]{$Y_0$} -- (demux.rpin 2) coordinate (Y0)
        (demux.brpin 3) node[left]{$Y_1$} -- (demux.rpin 3) coordinate (Y1)
        (demux.brpin 4) node[left]{$Y_2$} -- (demux.rpin 4) coordinate (Y2)
        (demux.brpin 5) node[left]{$Y_3$} -- (demux.rpin 5) coordinate (Y3)
        (demux.brpin 6) node[left]{$Y_4$} -- (demux.rpin 6) coordinate (Y4)
        (demux.brpin 7) node[left]{$Y_5$} -- (demux.rpin 7) coordinate (Y5)
        (demux.brpin 8) node[left]{$Y_6$} -- (demux.rpin 8) coordinate (Y6)
        (demux.brpin 9) node[left]{$Y_7$} -- (demux.rpin 9) coordinate (Y7)
        (demux.blpin 2) node[right]{$S_2$} -- (demux.lpin 2) coordinate (S2)
        (demux.blpin 3) node[right]{$S_1$} -- (demux.lpin 3) coordinate (S1)
        (demux.blpin 4) node[right]{$S_0$} -- (demux.lpin 4) coordinate (S0)

        (demux.brpin 2) node[ocirc,right]{}
        (demux.brpin 3) node[ocirc,right]{}
        (demux.brpin 4) node[ocirc,right]{}
        (demux.brpin 5) node[ocirc,right]{}
        (demux.brpin 6) node[ocirc,right]{}
        (demux.brpin 7) node[ocirc,right]{}
        (demux.brpin 8) node[ocirc,right]{}
        (demux.brpin 9) node[ocirc,right]{}

        (S2) coordinate(A) node[above]{A}
        (S1) coordinate(B) node[above]{B}
        (S0) coordinate(C) node[above]{C}
        (C) ++(0,-0.65) coordinate(D) node[above]{D}

        (Y0) ++(0.5,0.5) node[or port, number inputs=4, anchor=in 1, rotate=90](nand1){}
        (nand1.out) node[above]{X}
        (nand1.bin 1) node[ocirc,below]{}
        (nand1.bin 2) node[ocirc,below]{}
        (nand1.bin 3) node[ocirc,below]{}
        (nand1.bin 4) node[ocirc,below]{}
        
        (nand1.in 1) -- (nand1.bin 1|-Y1) node[circ](XY1){} -- (Y1)
        (nand1.in 2) -- (nand1.bin 2|-Y2) node[circ](XY2){} -- (Y2)
        (nand1.in 3) -- (nand1.bin 3|-Y5) node[circ](XY5){} -- (Y5)
        (nand1.in 4) -- (nand1.bin 4|-Y6) node[circ](XY6){} -- (Y6)
        
        (nand1.in 4) ++(0.45,0) node[or port, number inputs=4, anchor=in 1, rotate=90](nand2){}
        (nand2.out) node[above]{Y}
        (nand2.bin 1) node[ocirc,below]{}
        (nand2.bin 2) node[ocirc,below]{}
        (nand2.bin 3) node[ocirc,below]{}
        (nand2.bin 4) node[ocirc,below]{}

        (nand2.in 1) -- (nand2.in 1|-Y0) node[circ](YY0){} -- (Y0)
        (nand2.in 2) -- (nand2.in 2|-Y1) node[circ](YY1){} -- (XY1)
        (nand2.in 3) -- (nand2.in 3|-Y4) node[circ](YY4){} -- (Y4)
        (nand2.in 4) -- (nand2.in 4|-Y5) node[circ](YY5){} -- (XY5)

        (nand2.in 4) ++(0.45,0) node[or port, number inputs=4, anchor=in 1, rotate=90](nand3){}
        (nand3.out) node[above]{Z}
        (nand3.bin 1) node[ocirc,below]{}
        (nand3.bin 2) node[ocirc,below]{}
        (nand3.bin 3) node[ocirc,below]{}
        (nand3.bin 4) node[ocirc,below]{}
        (nand3.in 1) -- (nand3.in 1|-Y4) node[circ]{} -- (YY4)
        (nand3.in 2) -- (nand3.in 2|-Y5) node[circ]{} -- (YY5)
        (nand3.in 3) -- (nand3.in 3|-Y6) node[circ]{} -- (XY6)
        (nand3.in 4) -- (nand3.in 4|-Y7) node[circ]{} -- (Y7)

        (C) node[and port, rotate=180, anchor=in 1,scale=1.2](and){}
        (and.out) node[xor port, rotate=180, anchor=in 2,scale=1](xor){}
        (D) -- (D-|xor.in 1) -- (xor.in 1)
        (xor.out) node[left]{W}
        ;
    \end{circuitikz}
\end{center}

\pagebreak

\subsection{This circuit can be simplified to remove the 3:8 demux.  Draw the schematic for each output (W, X, Y and Z) using only the six basic Gates:}

\begin{table}[H]
    \centering
    \begin{tabular}{ccc:c|c:ccc}
        \toprule
        A&B&C&D&W&X&Y&Z\\
        \midrule
        0&0&0&0&0&0&1&0\\
        0&0&0&1&1&0&1&0\\
        \greyrule
        0&0&1&0&0&1&1&0\\
        0&0&1&1&1&1&1&0\\
        \greyrule
        0&1&0&0&0&1&0&0\\
        0&1&0&1&1&1&0&0\\
        \greyrule
        0&1&1&0&1&0&0&0\\
        0&1&1&1&0&0&0&0\\
        \bottomrule
    \end{tabular}
    \quad
    \begin{tabular}{ccc:c|c:ccc}
        \toprule
        A&B&C&D&W&X&Y&Z\\
        \midrule
        1&0&0&0&0&0&1&1\\
        1&0&0&1&1&0&1&1\\
        \greyrule
        1&0&1&0&0&1&1&1\\
        1&0&1&1&1&1&1&1\\
        \greyrule
        1&1&0&0&0&1&0&1\\
        1&1&0&1&1&1&0&1\\
        \greyrule
        1&1&1&0&1&0&0&1\\
        1&1&1&1&0&0&0&1\\
        \bottomrule
    \end{tabular}
\end{table}

Observing the truth table:
$$W=(B\cdot C)\oplus D$$
$$X=B\oplus C$$
$$Y=\overline{B}$$
$$Z=A$$

\begin{circuitikz}
    \draw
    (0,-8) node[left](A){A}
    (0,-6) node[left](B){B}
    (0,-4) node[left](C){C}
    (0,-2) node[left](D){D}

    (2,0) node[above](W){W}
    (4,0) node[above](X){X}
    (6,0) node[above](Y){Y}
    (8,0) node[above](Z){Z}
    
    (W) -- ++(0,-0.25) node[xor port, rotate=90, anchor=out](xor){}
    (D) -- (D-|xor.in 1) -- (xor.in 1)
    (xor.in 2) node[and port, rotate=90, anchor=out](and){}
    (C) -- (C-|and.in 1)node[circ](WC){} -- (and.in 1)
    (B) -- (B-|and.in 2)node[circ](WB){} -- (and.in 2)
    
    (X) -- ++(0,-0.25) node[xor port, rotate=90, anchor=out](xor){}
    (WC) -- (C-|xor.in 1) -- (xor.in 1)
    (WB) -- (B-|xor.in 2)node[circ](XB){} -- (xor.in 2)
    
    (Y) -- ++(0,-0.25) node[not port, rotate=90, anchor=out](not){}
    (XB) -- (B-|not.in) -- (not.in)
    
    (A) -- (A-|Z) -- (Z)


    ;
\end{circuitikz}

\end{document}
