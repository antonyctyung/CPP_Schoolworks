\documentclass{article}
\usepackage[utf8]{inputenc}
\usepackage{siunitx}
\usepackage[american,siunitx]{circuitikz}
\usepackage{amsmath}
\usepackage{svg}
\usepackage{booktabs}
\usepackage{float}
\renewcommand{\thesubsection}{\thesection.\alph{subsection}}
\newcommand{\equal}{=}

\title{ECE1101L\\Lab 9\\Voltage and Current Gain}
\author{Choi Tim Antony Yung}
\date{31 October 2019}

\begin{document}

\maketitle

\section*{Objective}
The objective of this lab is to measure and calculate gains G in parallel and series circuits.

\section*{Lab Result}
Lab Partner: Anthony Nursalim

%1
\section{A circuit as illustrated below was set up with a varying voltage source}
\begin{center}
    \begin{circuitikz}
        \draw 
            (1,2) 
            to[V_=$V_s$] (1,0)
            %(0,2) to[short,i>^=I] 
            (1,2) to[R=\SI{1}{\kilo\ohm}] (3,2)
            (3,2)to[R=\SI{2.2}{\kilo\ohm}]
            (5,2)to[R,l_=\SI{1.8}{\kilo\ohm} , v^=$V_2$] (5,0)--(1,0)
            ;
    \end{circuitikz}
\end{center}

\pagebreak

%1a
\subsection{The resistance of the three resistors in circuit were measured}
The \SI{1}{\kilo\ohm} resistor was measured to have a resistance of \SI{0.995}{\kilo\ohm}, the \SI{2}{\kilo\ohm} resistor was measured to have a resistance of \SI{2.21}{\kilo\ohm}, and the \SI{1.8}{\kilo\ohm} resistor was measured to have a resistance of \SI{1.796}{\kilo\ohm}.
\begin{table}[H]
\centering
    \begin{tabular}{@{}r r r@{}}
         \toprule
         measured & nominal & \%diff  \\
         \midrule
         \SI{0.995}{\kilo\ohm} & \SI{1}{\kilo\ohm} & -0.5\% \\
         \SI{2.21}{\kilo\ohm} & \SI{2.2}{\kilo\ohm} & 0.5\% \\
         \SI{1.796}{\kilo\ohm} & \SI{1.8}{\kilo\ohm} & -0.2\% \\
         \bottomrule
    \end{tabular}
\end{table}

%1b
\subsection{Voltage $V_2$ was measured as voltage $V_s$ was changed}
Below is the data set obtained.
\begin{center}
    \begin{tabular}{S[table-format=2.3] S[table-format=2.3]}
        \toprule
        {$V_s$ (V)} & {$V_2$ (V)}\\
        \midrule
        -4.97 & -1.786 \\
        -3.96 & -1.425 \\
        -3.50 & -1.258 \\
        -2.98 & -1.073 \\
        -2.03 & -0.728 \\
        -0.957 & -0.344 \\
        0.000 & 0.000 \\
        0.950 & 0.342 \\
        1.931 & 0.693 \\
        3.06 & 1.100 \\
        3.50 & 1.258 \\
        4.01 & 1.444 \\
        4.93 & 1.774 \\ 
        \bottomrule
    \end{tabular}
\end{center}

\pagebreak

%1c
\subsection{A graph is generated using the above data set}
The graph below illustrated $V_2$ as a function of $V_s$.
\begin{figure}[H]
\includesvg[width = 345pt]{chart}
\end{figure}

%1d
\subsection{A best fit line is generated}
The slope of the best fit line is 0.36.

%1e
\subsection{An equation for $V_2$ as a function of $V_s$ can be derived from the slope derived from the best fit line}
From the slope of the above chart, we can derive the gain $G = 0.36$. Thus, the relation between $V_2$ in volts and $V_s$ in volts is $V_2 = 0.36\cdot V_s$.

%1f
\subsection{The gain $G=\frac{V_2}{V_s}$ can be calculated as follow}
\begin{align}
    \intertext{From voltage divider rule:}
    V_2 &= \frac{R_2}{R_t} \cdot V_s\\
    \intertext{Since the resistors are in series:}
    R_t &= \SI{995}{\ohm} + \SI{2210}{\ohm} + \SI{1796}{\ohm} =\SI{5001}{\ohm}\\
    \intertext{Substituting $R_2 = \SI{1796}{\ohm}$ and (2) into (1):}
    V_2 &= \frac{1796}{5001} \cdot V_s\\
    G = \frac{V_2}{V_s} &= 0.359 \cdot \frac{V_s}{V_s} = 0.359
\end{align}
Thus, the gain $G = 0.359$

%1g
\subsection{The measured gain G derived from the chart is slightly higher than the gain G calculated}
\begin{table}[H]
\centering
    \begin{tabular}{@{} l r r r@{}}
         \toprule
         &measured & calculated & \%diff  \\
         \midrule
         G & 0.36 & 0.359 & 0.24\% \\
         \bottomrule
    \end{tabular}
\end{table}

%1h
\subsection{$V_2$ when $V_s = \SI{3.5}{\volt}$ can be predicted as follow}
Using the equation of gain $G=\frac{V_2}{V_s}$, given that $V_s = \SI{3.5}{volt}$ and $G = 0.36$, $V_2 = G\cdot V_s = 3.5 \times 0.36 = \SI{1.26}{\volt}$

%1i
\subsection{$V_2$ was measured to be \SI{1.258}{\volt} when $V_s = \SI{3.5}{\volt}$}

%1j
\subsection{The measured value of $V_2$ is slightly lower than the value calculated}
\begin{table}[H]
\centering
    \begin{tabular}{@{} l r r r@{}}
         \toprule
         &measured & calculated & \%diff  \\
         \midrule
         $V_2$ &\SI{1.258}{\volt} & \SI{1.26}{\volt} & -0.16\% \\
         \bottomrule
    \end{tabular}
\end{table}

%2
\section{A circuit as illustrated below was set up with a varying voltage source}
\begin{center}
    \begin{circuitikz}
        \draw 
            (0,2) 
            to[V_=$V_s$] (0,0) -- (6,0)
            (0,2)
            (0,2) -- (1,2) 
            to[R=$R_1$\equal\SI{4.7}{\kilo\ohm}] (1,0)
            (1,2) -- (3.5,2)
            to[R=$R_2$\equal\SI{10}{\kilo\ohm}] (3.5,0)
            (3.5,2) to[short,i<^=$I_2$] (6,2)
            to[R=$R_3$\equal\SI{10}{\kilo\ohm}] (6,0)
            ;
    \end{circuitikz}
\end{center}


%2a
\subsection{The resistance of the three resistors in circuit were measured}
The \SI{4.7}{\kilo\ohm} resistor was measured to have a resistance of \SI{4.63}{\kilo\ohm}, the first \SI{10}{\kilo\ohm} resistor was measured to have a resistance of \SI{9.95}{\kilo\ohm}, and the second \SI{10}{\kilo\ohm} resistor was measured to have a resistance of \SI{9.97}{\kilo\ohm}.
\begin{table}[H]
\centering
    \begin{tabular}{@{}l r r r@{}}
         \toprule
         &measured & nominal & \%diff  \\
         \midrule
         $R_1$&\SI{4.63}{\kilo\ohm} & \SI{4.7}{\kilo\ohm} & -1.5\% \\
         $R_2$&\SI{9.95}{\kilo\ohm} & \SI{10}{\kilo\ohm} & -0.5\% \\
         $R_3$&\SI{9.97}{\kilo\ohm} & \SI{10}{\kilo\ohm} & -0.3\% \\
         \bottomrule
    \end{tabular}
\end{table}

%2b
\subsection{Current $I_2$ was measured as voltage $V_s$ was changed}
Below is the data set obtained.
\begin{center}
    \begin{tabular}{S[table-format=2.3] S[table-format=2.6]}
        \toprule
        {$V_s$ (V)} & {$I_2$ (A)}\\
        \midrule
        -4.98 & 0.000493 \\
        -3.94 & 0.000390 \\
        -3.50 & 0.000346 \\
        -2.93 & 0.000290 \\
        -2.00 & 0.000197 \\
        -0.940 & 0.000092 \\
        0.000 & 0.000000 \\
        0.960 & -0.000095 \\
        1.97 & -0.000194 \\
        3.01 & -0.000298 \\
        3.50 & -0.000346 \\
        3.99 & -0.000395 \\
        4.96 & -0.000490 \\ 
        \bottomrule
    \end{tabular}
\end{center}

\pagebreak

%2c
\subsection{A graph is generated using the above data set}
The graph below illustrated $V_2$ as a function of $V_s$.
\begin{figure}[H]
\includesvg[width = 345pt]{chart2}
\end{figure}

%2d
\subsection{A best fit line is generated}
The slope of the best fit line is \SI{-9.89e-5}{ }.

%2e
\subsection{An equation for $I_2$ as a function of $V_s$ can be derived from the slope derived from the best fit line}
From the slope of the above chart, we can derive the gain $G =\frac{I_2}{V_s}=\SI{-9.89e-5}{ }$. Thus, the relation between $I_2$ in volts and $V_s$ in volts is $I_2 = \SI{-9.89e-5}{ }\cdot V_s$.

%2f
\subsection{The gain $G=\frac{I_2}{V_s}$ can be calculated as follow}
\begin{align}
    \intertext{From current divider rule:}
    I_2 &= -(\frac{R_t}{R_2} \cdot I_s)\\
    %\intertext{Since the resistors are in parallel:}
    %R_t &= \frac{1}{\frac{1}{\SI{995}{\ohm}} + \frac{1}{\SI{2210}{\ohm}} + \frac{1}{\SI{1796}{\ohm}}} =\SI{5001}{\ohm}\\
    \intertext{By Ohm's Law:}
    V_s &= R_t \cdot I_s\\
    I_s &= \frac{V_s}{R_t}
    \intertext{Substituting $R_2 = \SI{9970}{\ohm}$ and (7) into (5):}
    I_2 &= -(\frac{R_t\cdot V_s}{9970\cdot R_t}) = -\frac{V_s}{9970}\\
    G = \frac{I_2}{V_s} &= -(\frac{1\cdot V_s}{9970 \cdot V_s}) =-\frac{1}{9970} = \SI{-1.003e-4}{ }
\end{align}
Thus, the gain $G = \SI{-1.003e-4}{ }$

%2g
\subsection{The magnitude of the measured gain G derived from the chart is slightly lower than the gain G calculated}
\begin{table}[H]
\centering
    \begin{tabular}{@{} l r r r@{}}
         \toprule
         &measured & calculated & \%diff  \\
         \midrule
         G & \SI{-9.89e-5}{ } & \SI{-1.003e-4}{ } & -1.4\% \\
         \bottomrule
    \end{tabular}
\end{table}

%2h
\subsection{$I_2$ when $V_s = \SI{3.5}{\volt}$ can be predicted as follow}
Using the equation of gain $G=\frac{I_2}{V_s}$, given that $V_s = \SI{3.5}{\volt}$ and $G = \SI{-9.89e-5}{ }$, $I_2 = G\cdot V_s = 3.5 \times \SI{-9.89e-5}{ } = \SI{-0.000346}{\ampere}$

%2i
\subsection{$I_2$ was measured to be \SI{-0.000346}{\ampere} when $V_s = \SI{3.5}{\volt}$}

%2j
\subsection{The measured value of $I_2$ is slightly lower in magnitude than the value calculated}
\begin{table}[H]
\centering
    \begin{tabular}{@{} l r r r@{}}
         \toprule
         &measured & calculated & \%diff  \\
         \midrule
         $I_2$ &\SI{-0.000346}{\volt} & \SI{-0.00034615}{\volt} & -0.04\% \\
         \bottomrule
    \end{tabular}
\end{table}

%3
\section{A circuit as illustrated below was set up with a 5.00V voltage source}
\begin{center}
    \begin{circuitikz}
        \draw 
            (1,2) 
            to[V_=$V\equal\SI{5}{\volt}$] (1,0) -- (7,0)
            (1,2) to[R=$R_1$\equal\SI{1}{\kilo\ohm}] (3.5,2)
            to[R,v^=$V_1$,l_=$R_2$\equal\SI{1}{\kilo\ohm}] (3.5,0)
            (3.5,2) to [short, i=$I_2$] (4,2) to [R=$R_3$\equal\SI{1}{\kilo\ohm}, v_=$V_2$] (6.5,2) -- (7,2)
            to[open, v=$V_3$] (7,0)
            ;
    \end{circuitikz}
\end{center}

%3a
\subsection{The current $I_2$, the voltage $V_1$, $V_2$ and $V_3$ was calculated}
The current $I_2$ was calculated to be \SI{0}{\ampere}, the voltage $V_1$ was calculated to be \SI{2.50}{\volt}, the voltage $V_3$ was calculated to be the same, \SI{2.50}{\volt}, while the voltage $V_2$ was calculated to be \SI{0}{\volt}.

%3b
\subsection{The current $I_2$, the voltage $V_1$, $V_2$ and $V_3$ was measured}
The current $I_2$ was measured to be \SI{0}{\ampere}, the voltage $V_1$ was measured to be \SI{3.39}{\volt} and the voltage $V_2$ was measured to be the same, \SI{3.39}{\volt}.

%3c
\subsection{The calculated values and the measured values are mostly the same}
\begin{table}[H]
\centering
    \begin{tabular}{@{} l r r r@{}}
         \toprule
         &measured & calculated & \%diff  \\
         \midrule
            $V_1$ & 2.48 & 2.497479839 & -0.70\% \\
            $I_2$ & 0.000 & 0 & 0 \\
            $V_2$ & 0.000 & 0 & 0 \\
            $V_3$ & 2.48 & 2.497479839 & -0.70\% \\ 
         \bottomrule
    \end{tabular}
\end{table}

%3d
\subsection{There are no current flowing through $R_3$, the voltage across $R_3$ was the same, thus $V_3=V_1$}
As the current flow through where there is least resistance, as they arrive at the node connecting $R_1$ and $R_2$, they all chose to flow through $R_2$, as it have far less resistance than an open circuit, having a theoretically infinite resistance. As no current is flowing through $R_3$, the voltage across $R_3$ will be zero as well by Ohm's Law, meaning there are no potential difference before and after $R_3$, thus $V_1$, the voltage different across $R_2$ is the same as $V_3$, the one across the open circuit.

%4
\section{A circuit as illustrated below was set up with a 5.00V voltage source}
\begin{center}
    \begin{circuitikz}
        \draw 
            (0,3) 
            to[V_=$V\equal\SI{5}{\volt}$] (0,0) -- (5,0)
            (0,3)
            (0,3) to[short,i>^=I] (1,3) to[R=$R_1$\equal\SI{1}{\kilo\ohm}] (3.5,3)
            to[short, i=$I_1$] (3.5,2) to[R, l_=$R_2$\equal\SI{1}{\kilo\ohm}, v^=$V_1$] (3.5,0)
            (3.5,3) -- (5,3)
            to[short, i=$I_2$] (5,2) -- (5,0)
            ;
    \end{circuitikz}
\end{center}

%4a
\subsection{The current $I_1$ and $I_2$, and the voltage $V_1$ was calculated}
The current $I_1$ was calculated to be \SI{0}{\ampere}, The current $I_2$ was calculated to be \SI{0.005}{\ampere}, and the voltage $V_1$ was calculated to be \SI{0}{\volt}.

%4b
\subsection{The current $I_1$ and $I_2$, and the voltage $V_1$ was measured}
The current $I_1$ was measured to be \SI{0}{\ampere}, The current $I_2$ was measured to be \SI{0.00504}{\ampere}, and the voltage $V_1$ was measured to be \SI{0}{\volt}.

%4c
\subsection{The calculated values and the measured values are mostly the same}
\begin{table}[H]
\centering
    \begin{tabular}{@{} l r r r@{}}
         \toprule
         &measured & calculated & \%diff  \\
         \midrule
            $I_1$ & 0.000 & 0 & 0 \\
            $V_1$ & 0.000 & 0 & 0 \\
            $I_2$ & 0.00487 & 0.00504 & -3.28\% \\ 
         \bottomrule
    \end{tabular}
\end{table}

%4d
\subsection{There are no epc flowing through $R_2$}
As epc flow through where there is least resistance, as they arrive at the node connecting $R_1$ and $R_2$, they all chose to flow through the short circuit, as it, having an ideal resistance of zero, have far less resistance than $R_2$. Therefore, the current flowing through $R_2$, $I_1$, is zero as there are no epc flowing through $R_2$. Consuming no power from the epc, there are no potential difference across $R_2$, thus $V_1$ is zero.
\end{document}
