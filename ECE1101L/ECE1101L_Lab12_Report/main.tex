\documentclass{article}
\usepackage[utf8]{inputenc}
\usepackage{siunitx}
\usepackage[american,siunitx]{circuitikz}
\usepackage{amsmath}
\usepackage{svg}
\usepackage{booktabs}
\usepackage{float}
\usepackage{xparse, xfp}
\renewcommand{\thesubsection}{\thesection.\alph{subsection}}
\newcommand{\equal}{=}
\ExplSyntaxOn
\NewDocumentCommand{\defcon}{mm}
 {
  \cs_new:Npx #1 { \fp_eval:n { #2 } }
 }
\ExplSyntaxOff

\title{ECE1101L\\Lab 12\\Negative Gain Op Amp Circuits}
\author{Choi Tim Antony Yung}
\date{21 November 2019}

\begin{document}

\maketitle

\section*{Objective}
The objective of this lab is to explore the behavior of an operation amplifier (op amp) with a negative gain.

\section*{Lab Partner}
Antony Nursalim\\
Mario Molina

\pagebreak

%1
\section{An op amp circuit was set up as follow}
\begin{center}
    \begin{circuitikz}
        \draw 
            (0, 0) node[op amp] (opamp) {}
            (opamp.-) -- ++(-0.5,0) to[R,l_=$R_1\equal\SI{2}{\kilo\ohm}$] ++(-2,0) to[V,l_=$V_s\equal\SI{3}{\volt}$] ++(0,-2) node[ground]{} -- ++(2,0) coordinate (groundnode) -- (groundnode |- opamp.+) -- (opamp.+)
            (opamp.-) ++(-0.5,0) -- ++(0,1) coordinate (tempcoor) to[R=$R_2\equal\SI{4.7}{\kilo\ohm}$] (tempcoor -| opamp.out) -- (opamp.out) -- ++(1.5,0) coordinate (voplus) to[open, v=$V_o$] (voplus |- groundnode) -- (groundnode)
            
            % op amp pinout label
            (opamp.down) node[anchor=north, color=red] {7}
            (opamp.up) node[anchor=south, color=red] {4}
            (opamp.+) node[anchor=north west, color=red] {3}
            (opamp.-) node[anchor=south west, color=red] {2}
            (opamp.out) ++(-0.15,0) node[anchor=south east, color=red] {6}
            ;
    \end{circuitikz}
\end{center}

%1a
\subsection{The op amp pin numbers was labeled in the above diagram at the corresponding terminals}

%1b
\defcon{\conbba}{1.981}
\defcon{\conbbb}{1981}
\defcon{\conbbc}{-0.95}
\defcon{\conbbd}{4.620}
\defcon{\conbbe}{4620}
\defcon{\conbbf}{-1.70}
\subsection{The resistance of the two resistors in circuit were measured}
The \SI{2}{\kilo\ohm} resistor, $R_1$, was measured to have a resistance of \SI{\conbba}{\kilo\ohm}, and the \SI{4.7}{\kilo\ohm} resistor, $R_2$, was measured to have a resistance of \SI{\conbbd}{\kilo\ohm}.
\begin{table}[H]
\centering
    \begin{tabular}{@{}r r r r@{}}
         \toprule
         &measured & nominal & \%diff  \\
         \midrule
        $R_1$&\SI{\conbbb}{\ohm} & \SI{2000}{\ohm} & \conbbc\% \\
        $R_2$&\SI{\conbbe}{\ohm} & \SI{4700}{\ohm} & \conbbf\% \\ 
         \bottomrule
    \end{tabular}
\end{table}

%1c
\defcon{\conbc}{-6.80}
\subsection{$V_o$ was measured}
$V_o$ was measured to be \SI{\conbc}{\volt}.

%1d
\defcon{\conbd}{-7.00}
\subsection{$V_o$ was calculated}
$V_o = G\cdot V_s = -\frac{R_2}{R_1}\cdot V_s = -\frac{\SI{\conbbe}{\ohm}}{\SI{\conbbb}{\ohm}}\cdot \SI{3}{\volt} = \SI{\conbd}{\volt}$ \\
\\
$V_o$ is calculated to be \SI{\conbd}{\volt}.

%1e
\defcon{\conbe}{-2.81}
\subsection{The measured value of $V_o$ is lower than the calculated value}
\begin{table}[H]
\centering
    \begin{tabular}{@{}r r r r@{}}
         \toprule
         &measured & calculated & \%diff  \\
         \midrule
        $V_o$&\SI{\conbc}{\volt} & \SI{\conbd}{\volt} & \conbe\% \\
         \bottomrule
    \end{tabular}
\end{table}

%2
\section{An op amp circuit was set up as follow}
\begin{center}
    \begin{circuitikz}
        \draw 
            (0, 0) node[op amp] (opamp) {}
            (opamp.-) -- ++(-0.5,0) to[R,l_=$R_1\equal\SI{4.7}{\kilo\ohm}$] ++(-2.5,0) to [short, i<_=$I$] ++(-0.5,0) to[V,l_=$V_s\equal\SI{3}{\volt}$] ++(0,-2) node[ground]{} -- ++(2,0) coordinate (groundnode) -- (groundnode |- opamp.+) -- (opamp.+)
            (opamp.-) ++(-0.5,0) -- ++(0,1) coordinate (tempcoor) 
            (tempcoor -| opamp.out) to[vR,l_=$R_2$] (tempcoor)
            (tempcoor -| opamp.out) -- (opamp.out) -- ++(1.5,0) coordinate (voplus) to[open, v=$V_o$] (voplus |- groundnode) -- (groundnode)
            
            % op amp pinout label
            (opamp.down) node[anchor=north, color=red] {7}
            (opamp.up) node[anchor=south, color=red] {4}
            (opamp.+) node[anchor=north west, color=red] {3}
            (opamp.-) node[anchor=south west, color=red] {2}
            (opamp.out) ++(-0.15,0) node[anchor=south east, color=red] {6}
            ;
    \end{circuitikz}
\end{center}

%2a
\subsection{The op amp pin numbers was labeled in the above diagram at the corresponding terminals}

%2b
\defcon{\concba}{4.62}
\defcon{\concbb}{4620}
\defcon{\concbc}{-1.70}
\subsection{The resistance of the resistor $R_1$ in circuit were measured}
The \SI{4.7}{\kilo\ohm} resistor was measured to have a resistance of \SI{\concba}{\kilo\ohm}.
\begin{table}[H]
\centering
    \begin{tabular}{@{}r r r r@{}}
         \toprule
         &measured & nominal & \%diff  \\
         \midrule
        $R_1$&\SI{\concbb}{\ohm} & \SI{4700}{\ohm} & \concbc\% \\ 
         \bottomrule
    \end{tabular}
\end{table}

\pagebreak

%2c
\subsection{Voltage $V_o$ was measured as resistance $R_2$ was changed}
Below is the data set obtained.
\begin{center}
    \begin{tabular}{S[table-format=4.1] S[table-format=1.3]}
        \toprule
        {$R_2$ ($\Omega$)} & {$V_o$ (V)}\\
        \midrule
        0 & 0.028 \\
        2470 & -1.52 \\
        4960 & -3.07 \\
        7540 & -4.66 \\
        9870 & -6.11 \\ 
        \bottomrule
    \end{tabular}
\end{center}
The graph below illustrated the values of $V_o$ with respect of $R_2$.
\begin{figure}[H]
\includesvg[width = 345pt]{chart}
\end{figure}
$V_o$ seems to decrease at a constant rate as $R_2$ increase.

\pagebreak

%2d
\subsection{Current $I$ was measured as resistance $R_2$ was changed}
Below is the data set obtained.
\begin{center}
    \begin{tabular}{S[table-format=4.1] S[table-format=1.3]}
        \toprule
        {$R_2$ ($\Omega$)} & {$I$ (mA)}\\
        \midrule
        0 & 0.624 \\
        2470 & 0.624 \\
        4960 & 0.624 \\
        7540 & 0.624 \\
        9870 & 0.624 \\ 
        \bottomrule
    \end{tabular}
\end{center}
The graph below illustrated the values of $I$ with respect of $R_2$.
\begin{figure}[H]
\includesvg[width = 345pt]{chart2}
\end{figure}
$I$ seems to be constant as $R_2$ changes.

\pagebreak

%3
\section{An op amp circuit was set up as follow}
\begin{center}
    \begin{circuitikz}
        \draw 
            (0, 0) node[op amp] (opamp) {}
            (opamp.-) -- ++(-0.5,0) to[vR,l_=$R_B$] ++(-2,0) to[R,l_=$R_A\equal\SI{2}{\kilo\ohm}$] ++(-2,0) to[short, i<_=$I$] ++(-0.5,0) to[V,l_=$V_s\equal\SI{3}{\volt}$] ++(0,-2) node[ground]{} -- ++(2,0) coordinate (groundnode) -- (groundnode |- opamp.+) -- (opamp.+)
            (opamp.-) ++(-0.5,0) -- ++(0,1) coordinate (tempcoor) to[R=$R_2\equal\SI{4.7}{\kilo\ohm}$] (tempcoor -| opamp.out) -- (opamp.out) -- ++(1.5,0) coordinate (voplus) to[open, v=$V_o$] (voplus |- groundnode) -- (groundnode)
            
            % op amp pinout label
            (opamp.down) node[anchor=north, color=red] {7}
            (opamp.up) node[anchor=south, color=red] {4}
            (opamp.+) node[anchor=north west, color=red] {3}
            (opamp.-) node[anchor=south west, color=red] {2}
            (opamp.out) ++(-0.15,0) node[anchor=south east, color=red] {6}
            ;
    \end{circuitikz}
\end{center}

%3a
\subsection{The op amp pin numbers was labeled in the above diagram at the corresponding terminals}

%3b
\defcon{\condba}{4.62}
\defcon{\condbb}{4620}
\defcon{\condbc}{-1.70}
\defcon{\condbd}{1.987}
\defcon{\condbe}{1987}
\defcon{\condbf}{-0.65}
\subsection{The resistance of the two resistors in circuit were measured}
The \SI{4.7}{\kilo\ohm} resistor, $R_2$, was measured to have a resistance of \SI{\condba}{\kilo\ohm}, and the \SI{2}{\kilo\ohm} resistor, $R_A$, was measured to have a resistance of \SI{\condbd}{\kilo\ohm}.
\begin{table}[H]
\centering
    \begin{tabular}{@{}r r r r@{}}
         \toprule
         &measured & nominal & \%diff  \\
         \midrule
        $R_2$&\SI{\condbb}{\ohm} & \SI{4700}{\ohm} & \condbc\% \\
        $R_A$&\SI{\condbe}{\ohm} & \SI{2000}{\ohm} & \condbf\% \\ 
         \bottomrule
    \end{tabular}
\end{table}

\pagebreak

%3c
\subsection{Voltage $V_o$ was measured as resistance $R_1$ was changed}
Below is the data set obtained.
\begin{center}
    \begin{tabular}{S[table-format=4.1] S[table-format=1.3]}
        \toprule
        {$R_1$ ($\Omega$)} & {$V_o$ (V)}\\
        \midrule
        1988.2 & -4.28 \\
        4527 & -1.94 \\
        7017 & -1.24 \\
        9467 & -0.91 \\
        11857 & -0.74 \\ 
    \bottomrule
    \end{tabular}
\end{center}
The graph below illustrated the values of $V_o$ with respect of $R_1$.
\begin{figure}[H]
\includesvg[width = 345pt]{chart3}
\end{figure}
$V_o$ seems to increase at a decreasing rate as $R_B$ increase.

\pagebreak

%3d
\subsection{Current $I$ was measured as resistance $R_1$ was changed}
Below is the data set obtained.
\begin{center}
    \begin{tabular}{S[table-format=4.1] S[table-format=1.3]}
        \toprule
        {$R_1$ ($\Omega$)} & {$I$ (mA)}\\
        \midrule 
        1988.2 & 0.933 \\
        4527 & 0.426 \\
        7017 & 0.278 \\
        9467 & 0.207 \\
        11857 & 0.165 \\ 
        \bottomrule
    \end{tabular}
\end{center}
The graph below illustrated the values of $I$ with respect of $R_1$.
\begin{figure}[H]
\includesvg[width = 345pt]{chart4}
\end{figure}
$I$ seems to decrease at a decreasing rate as $R_1$ increase.

\end{document}