\documentclass{article}
\usepackage{graphicx}
\usepackage[dvipsnames,table]{xcolor}
\usepackage[utf8]{inputenc}
\usepackage{siunitx}
\usepackage[american,siunitx]{circuitikz}
\usepackage{amsmath}
\usepackage{svg}
\usepackage{booktabs}
\usepackage{float}
\usepackage{xparse, xfp}
\usepackage{multirow}
\usepackage{tikz}
\usepackage{karnaugh-map}
\usepackage{pdfpages}
\usepackage{hyperref}
\usepackage{adjustbox}
\hypersetup{
    colorlinks=true,
    linkcolor=blue,
    filecolor=magenta,      
    urlcolor=cyan,
}
\usetikzlibrary{calc}
%\usepackage[landscape]{geometry}
\renewcommand{\thesubsection}{\thesection.\alph{subsection}}
\newcommand{\equal}{=}
\newcommand{\greyrule}{\arrayrulecolor{black!30}\midrule\arrayrulecolor{black}}
\makeatletter
\newcommand\currcoor{\the\tikz@lastxsaved,\the\tikz@lastysaved}
\makeatother
\newcolumntype{:}{@{\hskip\tabcolsep\color{black!30}\vrule\hskip\tabcolsep}}

\ExplSyntaxOn
\NewExpandableDocumentCommand \groupify { O{\,\allowbreak} m m }
  { \jakob_groupify:nnn {#1} {#2} {#3} }
\cs_new:Npn \jakob_groupify:nnn #1 #2 #3
  { \__jakob_groupify_loop:nnw { 1 } {#2} #3 \q_recursion_tail {#1} \q_recursion_stop }
\cs_new:Npn \__jakob_groupify_loop:nnw #1 #2 #3
  {
    \quark_if_recursion_tail_stop:n {#3}
    \exp_not:n {#3}
    \int_compare:nNnTF {#1} = {#2}
      { \__jakob_groupify_sep:n }
      { \exp_args:Nf \__jakob_groupify_loop:nnw { \int_eval:n { #1+1 } } }
          {#2}
  }
\cs_new:Npn \__jakob_groupify_sep:n #1 #2 \q_recursion_tail #3
  {
    \tl_if_empty:nF {#2} { \exp_not:n {#3} }
    \__jakob_groupify_loop:nnw { 1 } {#1}
    #2 \q_recursion_tail {#3}
  }
\ExplSyntaxOff

\title{ECE 3301\\Introduction to Microcontrollers\\\,\\Assignment 5}
\author{Choi Tim Antony Yung}
\begin{document}
\maketitle

\thispagestyle{empty}
\setcounter{page}{0}

\newpage

\section{(25) List the different types of reset that can reset the PIC18F? Describe how the power‐on reset and the manual reset operate?}
\begin{itemize}
  \item Power-On Reset (POR)
  
  Upon power-up, a power-on reset pulse is generated internally whenever VDD rises above a certain threshold. This allows the PIC18F chip to start inthe initialized state when VDD is adequate for operation.\\
  
  \item Manual Reset
  
  A manual reset is performed by driving the $\overline{MCLR}$ pin to LOW via external circuitry. The internal on-chip circuitry connected to the $\overline{MCLR}$ pin ensures that the pin is LOW for at least \SI{2}{\micro\second} (minimum requirement for reset).\\

  \item Brown-out reset (BOR)
  \item Watchdog timer (WDT) reset
  \item RESET instruction
\end{itemize}

\newpage

\section{(20) Write a single assembly instruction to configure PIC18F:}

\subsection{all bits of Port C as inputs.}
\begin{verbatim}
  SETF TRISC
\end{verbatim}
\subsection{all bits of Port D as outputs.}
\begin{verbatim}
  CLRF TRISD
\end{verbatim}
\subsection{Bits 0 through 4 of Port B as inputs.}
For digital input:
\begin{verbatim}
  SETF TRISB
\end{verbatim}
For analog input:
\begin{verbatim}
  CLRF ADCON1
\end{verbatim}
\subsection{Bit 3 of Port A as output.}
\begin{verbatim}
  BCF TRISA, 3
\end{verbatim}

\newpage

\section{(20) Repeat Q2 using C code, more than one line of C code is accepted.}

\subsection{all bits of Port C as inputs.}
\begin{verbatim}
  TRISC = 0xFF;
\end{verbatim}
\subsection{all bits of Port D as outputs.}
\begin{verbatim}
  TRISD = 0;
\end{verbatim}
\subsection{Bits 0 through 4 of Port B as inputs.}
For digital input:
\begin{verbatim}
  TRISB &= 0xE0;    // mask off lower 5 bits and keep higher 3 bit intact
  TRISB += 0x1F;    // make lower 5 bit all 1 => bits 0 to 4 as inputs
\end{verbatim}
For analog input:
\begin{verbatim}
  ADCON1 &= 0xF0;   // mask off PCFG to make AN0 through AN12 analog input
\end{verbatim}
\subsection{Bit 3 of Port A as output.}
\begin{verbatim}
  TRISAbits.TRISA3 = 0;
\end{verbatim}


\newpage

\section{(5) What is the default clock frequency of the PIC18F4321?}
According to page 31 of \href{http://ww1.microchip.com/downloads/en/DeviceDoc/39689b.pdf}{the PIC18F4321 Datasheet}, the default output frequency of INTOSC on Reset is \SI{1}{\mega\hertz}

\newpage

\section{(5) In the PIC18F458, how many pins are designated as I/O port pins? how many pins are assigned to Vcc and GND?}
According to page 2 of \href{http://ww1.microchip.com/downloads/en/DeviceDoc/41159e.pdf}{the PIC18F458 Datasheet}, 
\begin{itemize}
  \item 33 pins are designated as I/O port pins; 
  \item 2 pins are assigned to Vcc; and 
  \item 2 pins are assigned to GND
\end{itemize}




\newpage

\section{(25) For the table below use the following assembly code to fill the table, note that lines 1, 2, 3, and 9 are dirctives not instructions.}
\begin{enumerate}
  \item \begin{verbatim}PRDL EQU   0x50        ; Assign an alias for GBR 0x50\end{verbatim}
  \item \begin{verbatim}PRDH EQU   0x51        ; Assign an alias for GBR 0x51\end{verbatim}
  \item \begin{verbatim}     ORG   0x100       ; Start program at program memory address 0x100\end{verbatim}
  \item \begin{verbatim}     MOVLW 0xF9        ; Move 0xF9 to the WREG\end{verbatim}
  \item \begin{verbatim}     MULLW B'11111001' ; Mutiply binary value B'11111001' by the content of WREG\end{verbatim}
  \item \begin{verbatim}     MOVFF PRODH, PRDL ; Move the high product output byte from SFR PRODH register to GPR 0x50\end{verbatim}
  \item \begin{verbatim}     MOVFF PRODL, PRDH ; Move the low product output byte from SFR PRODL register to GPR 0x51\end{verbatim}
  \item \begin{verbatim}HERE BRA   HERE        ; Create continuous loop\end{verbatim}
  \item \begin{verbatim}     END               ; End the program\end{verbatim}
\end{enumerate}

\adjustbox{max width=\textwidth}{%
\begin{tabular}{r|l|l|l|l}
  \toprule
  No.&Label&Opcode/  &Operand&Comments\\
          &&Directive&&\\
  \midrule
  1.&PRDL &EQU   &0x50        &; Assign an alias for GBR 0x50\\
  2.&PRDH &EQU   &0x51        &; Assign an alias for GBR 0x51\\
  3.&     &ORG   &0x100       &; Start program at program memory address 0x100\\
  4.&     &MOVLW &0xF9        &; Move 0xF9 to the WREG\\
  5.&     &MULLW &B'11111001' &; Mutiply binary value B'11111001' by the content of WREG\\
  6.&     &MOVFF &PRODH, PRDL &; Move the high product output byte from SFR PRODH register to GPR 0x50\\
  7.&     &MOVFF &PRODL, PRDH &; Move the low product output byte from SFR PRODL register to GPR 0x51\\
  8.&HERE &BRA   &HERE        &; Create continuous loop\\
  9.&     &END   &            &; End the program\\
  \bottomrule
\end{tabular}
}

\end{document}