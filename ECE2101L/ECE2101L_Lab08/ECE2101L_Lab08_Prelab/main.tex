\documentclass{article}
\usepackage[utf8]{inputenc}
\usepackage{siunitx}
\usepackage{graphics}
\usepackage[american,siunitx]{circuitikz}
\usepackage{amsmath}
\usepackage{svg}
\usepackage{booktabs}
\usepackage{float}
\usepackage{xparse, xfp}
\usepackage{graphicx} 
\usepackage{steinmetz}
\usepackage{hyperref}
\hypersetup{
    colorlinks=true,
    linkcolor=blue,
    filecolor=magenta,      
    urlcolor=blue,
}
\renewcommand{\thesubsubsection}{\thesubsection.\alph{subsubsection}}
\newcommand{\equal}{=}
\ExplSyntaxOn
\NewDocumentCommand{\defcon}{mm}
 {
  \cs_new:Npx #1 { \fp_eval:n { #2 } }
 }
\ExplSyntaxOff

\title{ECE2101L\\Electrical Circuit Analysis II Laboratory\\\,\\Lab 8\\ Op Amp in AC Circuits\\\,\\Prelab\\}
\author{Choi Tim Antony Yung}
%\author{Choi Tim Antony Yung\\\,\\Willis Nguyen\\Phineas Cozmiuc}
\date{23 March 2020}

\begin{document}

\clearpage\maketitle
\thispagestyle{empty}
\newpage
\setcounter{page}{1}

\section{Op amp LM 741 }
\subsection{Pinout and circuit diagram of LM741}
\begin{center}
    \begin{circuitikz}
        \draw 
            (0,0) node[dipchip,num pins=8,font=\tiny](opamp){LM741}
            (opamp.pin 1) node[left]{Offset Null}
            (opamp.pin 2) node[left]{Inverting Signal Input $V_-$}
            (opamp.pin 3) node[left]{Non-Inverting Signal Input $V_+$}
            (opamp.pin 4) node[left]{Negative Supply Voltage $V_{ee}$}
            (opamp.pin 5) node[right]{Offset Null}
            (opamp.pin 6) node[right]{Amplified Signal Output}
            (opamp.pin 7) node[right]{$V_{cc}$ Positive Supply Voltage}
            (opamp.pin 8) node[right]{No Connect}
            
            
            (0,4) node[op amp, yscale=-1] (opamp) {}
            (opamp.down) to[short] ++(0,0.4) node[vcc]{$V_{cc}$} %Vcc
            (opamp.up) to[short] ++(0,-0.4) node[vee]{$V_{ee}$} %Vee
            
            % op amp pinout label
            (opamp.down) to[open] ++ (0,0.15) node[anchor=south east, color=red] {7}
            (opamp.up) ++ (0,-0.15) node[anchor=north east, color=red] {4}
            (opamp.+) node[anchor=north west, color=red] {3}
            (opamp.-) node[anchor=south west, color=red] {2}
            (opamp.out) ++(-0.15,0) node[anchor=south east, color=red] {6}
            ;
    \end{circuitikz}
\end{center}

\subsection{Specification of LM741}
\href{http://www.ti.com/lit/ds/symlink/lm741.pdf}{Texas Instrument's LM741 Datasheet} section Absolute Maximum Rating is referenced for the information below:
\subsubsection{Supply voltage rating: $\pm\SI{22}{\volt}$}
\subsubsection{Input voltage rating: $\pm\SI{15}{\volt}$}
\subsubsection{Differential Input voltage rating: $\pm\SI{30}{\volt}$}
\,\\


\section{Voltage and current relation in op amp circuit}
\begin{center}
    \begin{circuitikz}
        \draw 
            (0, 0) node[op amp, yscale=-1] (opamp) {}
            (opamp.+) to[short] ++(-0.01,0) -- ++(-2.99,0) to[V,l_=$v_s(t)$] ++(0,-5) coordinate (groundnode) node[ground]{}
            (opamp.-) to[short] ++(-0.01,0) -- ++(-0.34,0) -- ++(0,-2) coordinate (vnnode)
            -- (vnnode -| opamp.-) to[R,l_=$R_f$] (vnnode -| opamp.out)
            (groundnode) -- (groundnode -| vnnode) to[R=$R_g$] (vnnode)
            (opamp.out) -- ++(0.5,0) coordinate (outnode)
            (vnnode -| opamp.out) -- ++(0.5,0) to[short,i<=$i(t)$] (outnode) -- ++(1.5,0) coordinate (vnaughtplusnode)
            (groundnode -| vnnode) -- (groundnode -| vnaughtplusnode) to[open, v^<=$v_o(t)$] (vnaughtplusnode)
            
            % op amp pinout label
            (opamp.down) to[open] ++ (0,0.15) node[anchor=south east, color=red] {7}
            (opamp.up) ++ (0,-0.15) node[anchor=north east, color=red] {4}
            (opamp.+) node[anchor=north west, color=red] {3}
            (opamp.-) node[anchor=south west, color=red] {2}
            (opamp.out) ++(-0.15,0) node[anchor=south east, color=red] {6}
            ;
    \end{circuitikz}
\end{center}
\subsection{Output voltage $v_o$ and current $i$}
Assuming ideal op amp, no current is flowing into inverting input and there is no voltage difference between inverting and non-inverting input, therefore i is both the current flowing across $R_f$ and across $R_g$ and voltage at inverting input is $v_s$. Then:
$$i=\frac{v_s}{R_g}$$
$$v_o=iR=\frac{v_s}{R_g}(R_f+R_g)=(\frac{R_f}{R_g}+1)v_s$$

\subsection{Gain}
$$G=\frac{v_o}{v_s}=(\frac{R_f}{R_g}+1)\frac{v_s}{v_s}=\frac{R_f}{R_g}+1$$


\end{document}