\documentclass{article}
\usepackage[margin=1in]{geometry}
\usepackage[utf8]{inputenc}
\usepackage{listings}
\usepackage{hyperref}
\usepackage{graphicx}
\usepackage{float}
\usepackage{xcolor}
\usepackage{indentfirst}

\definecolor{codegreen}{rgb}{0,0.7,0}
\definecolor{codegray}{rgb}{0.5,0.5,0.5}
\definecolor{codepurple}{rgb}{0.58,0,0.82}
\definecolor{backcolour}{rgb}{0.95,0.95,0.92}

\lstdefinestyle{mystyle}{
    backgroundcolor=\color{backcolour},   
    commentstyle=\color{codegreen},
    keywordstyle=\color{magenta},
    numberstyle=\tiny\color{codegray},
    stringstyle=\color{codepurple},
    basicstyle=\ttfamily\footnotesize,
    breakatwhitespace=false,         
    breaklines=true,                 
    captionpos=b,                    
    keepspaces=true,                 
    numbers=left,                    
    numbersep=5pt,                  
    showspaces=false,                
    showstringspaces=false,
    showtabs=false,                  
    tabsize=2
}

\lstset{style=mystyle}

\title{ECE 4310\\Operating Systems for Embedded Application\\\,\\Extra Credit 1}
\author{Choi Tim Antony Yung}
\date{May 13, 2021}
\begin{document}
\maketitle

\thispagestyle{empty}
\setcounter{page}{0}

\newpage

\section{What is the Mars 2020 team looking to implement in order to increase the productivity of the Mars 2020 rover?}

The Mars 2020 team is looking to implement an onboard scheduler to:
\begin{itemize}
  \item depending on status of the rover not known to remote operators, plan extra activity
 when more resources than planned is available without waiting for communication
  \item create tasks on-board to reduce remote operator's time spent on doing that
\end{itemize}

\section{Explain the 3 main phases of the Mars 2020 rover’s mission.}
\begin{enumerate}
  \item Cruise: the spacecraft containing the rover travels from Earth to Mars
  \item Entry Descent and Landing: the rover is being landed on Mars
  \item Surface: the rover drives on Mars and perform tasks
\end{enumerate}

\section{Explain the hardware used in the onboard flight computer and operating system being utilized.}
Hardware: BAE Rad750 single board computer
\begin{itemize}
  \item PowerPC architecture
  \item 128 MB volatile DRAM
  \item 133 MHz clock speed
  \item 4 GB of NAND non-volatile memory on separate card
\end{itemize}

Operating system: VxWorks\texttrademark
\begin{itemize}
  \item RTOS
  \item written in C
  \item flight software decomposed into different module each running different VxWorks tasks
  \item modules communicate with each other via inter-process communication (IPC)
  \item priority queue that can be individually enabled/disabled to manage IPC messages
\end{itemize}

\section{What is a martian \textit{sol}?  Explain how it compares to that of time on earth?}
A martian \textit{sol} is a solar day on Mars, which is equivalent to approximately 24 hours and 40 minutes

\section{How are timelines, and their constraints, being used in this algorithm for planning of activities in the Mars 2020 rover?  Explain thoroughly.}
Timelines are used to represent the projected states of the rover performing planned activities.
Each timelines represents a state of the rover and a constraint on that timeline is the range of value that state should be in.
An impact is a change in the value of the state the timeline is recording.
When an activity is being planned, the algorithm goes through each timeline and make sure to start
the activity at a time that does not bring the rover to a state that it should not be in by calculating
its impacts on the timelines and whether or not it will violate the constraints of each timelines.

\section{What algorithm are they looking at implementing for scheduling on the Mars 2020 rover and how will it help increase productivity of the rover?  Explain thoroughly.}
A greedy algorithm is used to schedule activities. It add activities from highest priority to lowest, with all mandatory activities having a higher priority than all optional activities.
For each activities, it will attempt to find an intersection of valid start time in all timelines
such that no constraints in any timelines are violated.
If there exist such a start time, the activity will then be added and new impacts and possibly new constraints will be added to the timelines.
By using this algorithm, it can schedule the activities such that more activities can be performed 
concurrently given that there are not violation on the constraints.

\section{Explain in your own words whether you think this scheduling algorithm may or may not be effective in increasing the productivity of the Mars 2020 rover?}
I believe that this scheduling algorithm is effective in increasing the productivity of the Mars 2020 rover.
One of the goals of this scheduler is to allow for dynamic rescheduling of the activities on the rover without reaching to remote operators.
The time delay in communication may produce missed opportunity in which an activity may be
able to be squeezed in when performed activities uses less resources/ take less time etc.
This scheduling algorithm helps mitigate that by dynamically rescheduling depending on the
state of the rovers and new estimation of the state of the rover. 


\end{document}