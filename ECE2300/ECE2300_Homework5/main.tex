\documentclass{article}
\usepackage[table]{xcolor}
\usepackage[utf8]{inputenc}
\usepackage{siunitx}
\usepackage[american,siunitx]{circuitikz}
\usepackage{amsmath}
\usepackage{svg}
\usepackage{booktabs}
\usepackage{float}
\usepackage{xparse, xfp}
%\usepackage[landscape]{geometry}
\renewcommand{\thesubsection}{\thesection.\alph{subsection}}
\newcommand{\equal}{=}
\newcommand{\greyrule}{\arrayrulecolor{black!30}\midrule\arrayrulecolor{black}}
\newcolumntype{:}{@{\hskip\tabcolsep\color{black!30}\vrule\hskip\tabcolsep}}

\title{ECE 2300\\Digital Logic Design\\\,\\Homework 5}
\author{Choi Tim Antony Yung}
\begin{document}
\maketitle

\thispagestyle{empty}
\setcounter{page}{0}

\newpage

\section{Using an 8:1 mux, create a circuit to generate}
$$f(ABCD)=\overline{W}\cdot\overline{Z}+\overline{X}\cdot\overline{Y}\cdot Z+X\cdot Y\cdot Z$$

\begin{table}[H]
    \centering
    \begin{tabular}{ccc:c|c}
        \toprule
        W&X&Y&Z&$f$\\
        \midrule
        0&0&0&0&1\\
        0&0&0&1&1\\
        \greyrule
        0&0&1&0&1\\
        0&0&1&1&0\\
        \greyrule
        0&1&0&0&1\\
        0&1&0&1&0\\
        \greyrule
        0&1&1&0&1\\
        0&1&1&1&1\\
        \bottomrule
    \end{tabular}
    \quad
    \begin{tabular}{ccc:c|c}
        \toprule
        W&X&Y&Z&$f$\\
        \midrule
        1&0&0&0&0\\
        1&0&0&1&1\\
        \greyrule
        1&0&1&0&0\\
        1&0&1&1&0\\
        \greyrule
        1&1&0&0&0\\
        1&1&0&1&0\\
        \greyrule
        1&1&1&0&0\\
        1&1&1&1&1\\
        \bottomrule
    \end{tabular}
\end{table}
\,\\
\begin{center}
    \begin{circuitikz}
        \tikzset{mux/.style={muxdemux,muxdemux def={Lh= 10,Rh= 7,NL=10},no input leads}}
        \draw
        (0,0) node[mux](mux){}

        (mux.blpin 2) node[right]{$d_0$} -- (mux.lpin 2) coordinate (d0)
        (mux.blpin 3) node[right]{$d_1$} -- (mux.lpin 3) coordinate (d1)
        (mux.blpin 4) node[right]{$d_2$} -- (mux.lpin 4) coordinate (d2)
        (mux.blpin 5) node[right]{$d_3$} -- (mux.lpin 5) coordinate (d3)
        (mux.blpin 6) node[right]{$d_4$} -- (mux.lpin 6) coordinate (d4)
        (mux.blpin 7) node[right]{$d_5$} -- (mux.lpin 7) coordinate (d5)
        (mux.blpin 8) node[right]{$d_6$} -- (mux.lpin 8) coordinate (d6)
        (mux.blpin 9) node[right]{$d_7$} -- (mux.lpin 9) coordinate (d7)

        (mux.bbpin 1) node[above]{\small{$S_2$}} -- (mux.bpin 1) coordinate (S2)
        (mux.bbpin 2) node[above]{\small{$S_1$}} -- (mux.bpin 2) coordinate (S1)
        (mux.bbpin 3) node[above]{\small{$S_0$}} -- (mux.bpin 3) coordinate (S0)

        (mux.brpin 1) node[left]{$Y$} -- (mux.rpin 1) coordinate (Y)

        (d0) node[above]{1}
        (d1) node[above]{$\overline{Z}$}
        (d2) node[above]{$\overline{Z}$}
        (d3) node[above]{1}
        (d4) node[above]{Z}
        (d5) node[above]{0}
        (d6) node[above]{0}
        (d7) node[above]{Z}

        (S2) node[below]{W}
        (S1) node[below]{X}
        (S0) node[below]{Y}

        (Y) node[right]{$f$}

        (d0) -- ++(-0.5,0) node[vcc]{$V_{cc}$}
        (d1) ++(-0.5,0) node[jump crossing](xing1){}
        (d2) ++(-0.5,0) node[jump crossing](xing2){}
        (d3) -- ++(-0.5,0) -- (xing2.south)
        (xing2.north) -- (xing1.south)
        (d0) ++(-0.5,0) -- (xing1.north)
        (xing1.east) -- (d1)
        (xing2.east) -- (d2)
        (d6) -- ++(-0.5,0) -- ++(0,-0.2) node[tlground](gnd){}
        (d5) -- ++(-0.5,0) -- (d6-|gnd)
        (d7) -- ++(-1,0) coordinate(Z) -- (Z|-d4) node[not port, scale=0.75, anchor=in, rotate=90](not){} -- (d4)
        (not.out) -- (not.out|-d2) -- (xing2.west)
        (not.out|-d2) -- (not.out|-d1) -- (xing1.west)
        (Z) -- ++(0,-1) node[below]{Z}
        ;
    \end{circuitikz}
\end{center}

\pagebreak

\section{Using two 4:1 muxes, generate the $B_O$ (Borrow-Out) and $C_O$ (Carry-Out) outputs for a 1-bit full subtractor and 1-bit full adder respectively where X is the minuend, Y is the subtrahend or Y and Y are the addends and bc is the borrow-in or carry-in.}
$$B_O=\overline{X}\cdot b_c+\overline{X}\cdot Y+Y\cdot b_c$$
$$C_O=X\cdot Y+X\cdot b_c+Y\cdot b_c$$

\begin{table}[H]
    \centering
    \begin{tabular}{cc:c|cc}
        \toprule
        X&Y&$b_c$&$B_O$&$C_O$\\
        \midrule
        0&0&0&0&0\\
        0&0&1&1&0\\
        \greyrule
        0&1&0&1&0\\
        0&1&1&1&1\\
        \bottomrule
    \end{tabular}
    \quad
    \begin{tabular}{cc:c|cc}
        \toprule
        X&Y&$b_c$&$B_O$&$C_O$\\
        \midrule
        1&0&0&0&0\\
        1&0&1&0&1\\
        \greyrule
        1&1&0&0&1\\
        1&1&1&1&1\\
        \bottomrule
    \end{tabular}
\end{table}

\,\\
\begin{center}
    \begin{circuitikz}
        \tikzset{mux/.style={muxdemux,muxdemux def={Lh= 6,Rh= 4,NL=6,NB=2,w=2},no input leads}}
        \draw
        (0,0) node[mux](mux){}

        (mux.blpin 2) node[right]{$d_0$} -- (mux.lpin 2) coordinate (1d0)
        (mux.blpin 3) node[right]{$d_1$} -- (mux.lpin 3) coordinate (1d1)
        (mux.blpin 4) node[right]{$d_2$} -- (mux.lpin 4) coordinate (1d2)
        (mux.blpin 5) node[right]{$d_3$} -- (mux.lpin 5) coordinate (1d3)

        (mux.bbpin 1) node[above]{\small{$S_1$}} -- (mux.bpin 1) coordinate (1S1)
        (mux.bbpin 2) node[above]{\small{$S_0$}} -- (mux.bpin 2) coordinate (1S0)

        (mux.brpin 1) node[left]{$Y$} -- (mux.rpin 1) coordinate (1Y)

        (4,0) node[mux](mux2){}

        (mux2.blpin 2) node[right]{$d_0$} -- (mux2.lpin 2) coordinate (2d0)
        (mux2.blpin 3) node[right]{$d_1$} -- (mux2.lpin 3) coordinate (2d1)
        (mux2.blpin 4) node[right]{$d_2$} -- (mux2.lpin 4) coordinate (2d2)
        (mux2.blpin 5) node[right]{$d_3$} -- (mux2.lpin 5) coordinate (2d3)

        (mux2.bbpin 1) node[above]{\small{$S_1$}} -- (mux2.bpin 1) coordinate (2S1)
        (mux2.bbpin 2) node[above]{\small{$S_0$}} -- (mux2.bpin 2) coordinate (2S0)

        (mux2.brpin 1) node[left]{$Y$} -- (mux2.rpin 1) coordinate (2Y)

        (1S1-|1S0) node[jump crossing](xing){}
        (xing.north)-- (1S0)
        (xing.east) -- (2S1)
        (xing.west) -- (1S1)
        (xing.south) -- ++(0,-0.25) node[circ](cntk){}
        (2S0) -- (2S0|-cntk) -- (mux.lpin 1|-cntk) node[left]{Y}
        (1S1) node[circ]{} -- (mux.lpin 1|-2S1) node[left]{X}

        (1Y) node[right]{$B_O$}
        (2Y) node[right]{$C_O$}

%        (1d0) node[above]{$b_c$}
        (1d1) node[left]{1}
        (1d2) node[above]{0}
%        (1d3) node[above]{$b_c$}
        
        (2d0) node[above]{0}
        (2d1) node[above]{$b_c$}
%        (2d2) node[above]{$b_c$}
        (2d3) node[above]{1}
        
        (1d1) node[vcc]{}

        (2d0) -- ++(-0.5,0) node[ground]{}
        (2d1) -- (2d2)
        (2d3) -- ++(-0.5,0) node[vcc]{}

        (1d0) -- ++(-0.5,0) node[above](bc){$b_c$} -- (bc|-1d3) -- (1d3)
        (1d2) -- ++(0,-0.2) node[tlground]{}
        ;
    \end{circuitikz}
\end{center}

\pagebreak

\section{Using an 8:1 mux, create a circuit to generate the segment $g$ outputs for a 7-segment display where $g=\overline{B}\cdot \overline{D}+C\cdot \overline{D}$.}
\begin{table}[H]
    \centering
    \begin{tabular}{ccc:c|c}
        \toprule
        A&B&C&D&$g$\\
        \midrule
        0&0&0&0&1\\
        0&0&0&1&0\\
        \greyrule
        0&0&1&0&1\\
        0&0&1&1&0\\
        \greyrule
        0&1&0&0&0\\
        0&1&0&1&0\\
        \greyrule
        0&1&1&0&1\\
        0&1&1&1&0\\
        \bottomrule
    \end{tabular}
    \quad
    \begin{tabular}{ccc:c|c}
        \toprule
        A&B&C&D&$g$\\
        \midrule
        1&0&0&0&1\\
        1&0&0&1&0\\
        \greyrule
        1&0&1&0&1\\
        1&0&1&1&0\\
        \greyrule
        1&1&0&1&0\\
        1&1&0&0&0\\
        \greyrule
        1&1&1&0&1\\
        1&1&1&1&0\\
        \bottomrule
    \end{tabular}
\end{table}
\,\\
\begin{center}
    \begin{circuitikz}
        \tikzset{mux/.style={muxdemux,muxdemux def={Lh= 10,Rh= 7,NL=10},no input leads}}
        \draw
        (0,0) node[mux](mux){}

        (mux.blpin 2) node[right]{$d_0$} -- (mux.lpin 2) coordinate (d0)
        (mux.blpin 3) node[right]{$d_1$} -- (mux.lpin 3) coordinate (d1)
        (mux.blpin 4) node[right]{$d_2$} -- (mux.lpin 4) coordinate (d2)
        (mux.blpin 5) node[right]{$d_3$} -- (mux.lpin 5) coordinate (d3)
        (mux.blpin 6) node[right]{$d_4$} -- (mux.lpin 6) coordinate (d4)
        (mux.blpin 7) node[right]{$d_5$} -- (mux.lpin 7) coordinate (d5)
        (mux.blpin 8) node[right]{$d_6$} -- (mux.lpin 8) coordinate (d6)
        (mux.blpin 9) node[right]{$d_7$} -- (mux.lpin 9) coordinate (d7)

        (mux.bbpin 1) node[above]{\small{$S_2$}} -- (mux.bpin 1) coordinate (S2)
        (mux.bbpin 2) node[above]{\small{$S_1$}} -- (mux.bpin 2) coordinate (S1)
        (mux.bbpin 3) node[above]{\small{$S_0$}} -- (mux.bpin 3) coordinate (S0)

        (mux.brpin 1) node[left]{$Y$} -- (mux.rpin 1) coordinate (Y)

        (d0) node[above]{$\overline{D}$}
        (d1) node[above]{$\overline{D}$}
        (d2) node[above]{0}
        (d3) node[above]{$\overline{D}$}
        (d4) node[above]{$\overline{D}$}
        (d5) node[above]{$\overline{D}$}
        (d6) node[above]{0}
        (d7) node[above]{$\overline{D}$}

        (S2) node[below]{A}
        (S1) node[below]{B}
        (S0) node[below]{C}

        (Y) node[right]{$g$}

        (d7) -- ++(-1,0) node[not port,anchor=out, rotate=90](not){}
        (not.in) node[below]{D}
        (d6) -- ++(-0.5,0) -- ++(0,-0.2) node[tlground](gnd){}
        (d2) -- ++(-0.5,0) -- ++(0,-0.2) node[tlground](gnd){}
        (d0) -- ++(-1,0) -- (not.out)
        (d1) -- ++(-1,0)
        (d3) -- ++(-1,0)
        (d4) -- ++(-1,0)
        (d5) -- ++(-1,0)
        ;
    \end{circuitikz}
\end{center}

\pagebreak

\end{document}
