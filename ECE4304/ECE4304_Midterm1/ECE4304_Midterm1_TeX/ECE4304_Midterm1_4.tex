\documentclass{article}
\usepackage[margin=1in]{geometry}
\usepackage{graphicx}
\usepackage[dvipsnames,table]{xcolor}
\usepackage[utf8]{inputenc}
\usepackage{siunitx}
\usepackage[american,siunitx]{circuitikz}
\usepackage{amsmath}
\usepackage{svg}
\usepackage{booktabs}
\usepackage{float}
\usepackage{xparse, xfp}
\usepackage{multirow}
\usepackage{tikz}
\usepackage{karnaugh-map}
\usepackage{pdfpages}
\usepackage{pdflscape}
\usepackage{hyperref}
\usepackage{adjustbox}

\newcommand{\overbar}[1]{\mkern 1.5mu\overline{\mkern-1.5mu#1\mkern-1.5mu}\mkern 1.5mu}
\hypersetup{
    colorlinks=true,
    linkcolor=blue,
    filecolor=magenta,      
    urlcolor=cyan,
}
\usepackage{caption} 

\makeatletter
\ctikzset{lx/.code args={#1 and #2}{ 
  \pgfkeys{/tikz/circuitikz/bipole/label/name=\parbox{1cm}{\centering #1  \\ #2}}
    \ctikzsetvalof{bipole/label/unit}{}
    \ifpgf@circ@siunitx 
        \pgf@circ@handleSI{#2}
        \ifpgf@circ@siunitx@res 
            \edef\pgf@temp{\pgf@circ@handleSI@val}
            \pgfkeyslet{/tikz/circuitikz/bipole/label/name}{\pgf@temp}
            \edef\pgf@temp{\pgf@circ@handleSI@unit}
            \pgfkeyslet{/tikz/circuitikz/bipole/label/unit}{\pgf@temp}
        \else
        \fi
    \else
    \fi
}}

\ctikzset{lx^/.style args={#1 and #2}{ 
    lx=#2 and #1,
    \circuitikzbasekey/bipole/label/position=90 } 
}

\ctikzset{lx_/.style args={#1 and #2}{ 
    lx=#1 and #2,
    \circuitikzbasekey/bipole/label/position=-90 } 
}
\makeatother

\captionsetup[table]{skip=10pt}

\usetikzlibrary{calc, automata, positioning}
%\usepackage[landscape]{geometry}
\renewcommand{\thesubsection}{\thesection.\alph{subsection}}
\newcommand{\equal}{=}
\newcommand{\greyrule}{\arrayrulecolor{black!30}\midrule\arrayrulecolor{black}}
\makeatletter
\newcommand\currcoor{\the\tikz@lastxsaved,\the\tikz@lastysaved}
\makeatother
\newcolumntype{:}{@{\hskip\tabcolsep\color{black!30}\vrule\hskip\tabcolsep}}

\ExplSyntaxOn
\NewExpandableDocumentCommand \groupify { O{\,\allowbreak} m m }
  { \jakob_groupify:nnn {#1} {#2} {#3} }
\cs_new:Npn \jakob_groupify:nnn #1 #2 #3
  { \__jakob_groupify_loop:nnw { 1 } {#2} #3 \q_recursion_tail {#1} \q_recursion_stop }
\cs_new:Npn \__jakob_groupify_loop:nnw #1 #2 #3
  {
    \quark_if_recursion_tail_stop:n {#3}
    \exp_not:n {#3}
    \int_compare:nNnTF {#1} = {#2}
      { \__jakob_groupify_sep:n }
      { \exp_args:Nf \__jakob_groupify_loop:nnw { \int_eval:n { #1+1 } } }
          {#2}
  }
\cs_new:Npn \__jakob_groupify_sep:n #1 #2 \q_recursion_tail #3
  {
    \tl_if_empty:nF {#2} { \exp_not:n {#3} }
    \__jakob_groupify_loop:nnw { 1 } {#1}
    #2 \q_recursion_tail {#3}
  }
\ExplSyntaxOff

% \title{ECE 3300\\Digital Circuit Design Using Verilog\\\,\\Exercise 3}
% \author{Choi Tim Antony Yung}
\begin{document}
% \maketitle

% \thispagestyle{empty}
% \setcounter{page}{0}
% \newpage

\pagenumbering{gobble}


\section*{Truth Table}
\begin{table}[H]
  \centering
  \begin{tabular}{ccccc|c}
    \toprule
    $A$ & $B$ & $C$ & $D$ & $E$ & $maj$ \\
    \midrule
    0   & 0   & 0   & 0   & 0   & 0     \\
    0   & 0   & 0   & 0   & 1   & 0     \\
    0   & 0   & 0   & 1   & 0   & 0     \\
    0   & 0   & 0   & 1   & 1   & 0     \\
    0   & 0   & 1   & 0   & 0   & 0     \\
    0   & 0   & 1   & 0   & 1   & 0     \\
    0   & 0   & 1   & 1   & 0   & 0     \\
    0   & 0   & 1   & 1   & 1   & 1     \\
    0   & 1   & 0   & 0   & 0   & 0     \\
    0   & 1   & 0   & 0   & 1   & 0     \\
    0   & 1   & 0   & 1   & 0   & 0     \\
    0   & 1   & 0   & 1   & 1   & 1     \\
    0   & 1   & 1   & 0   & 0   & 0     \\
    0   & 1   & 1   & 0   & 1   & 1     \\
    0   & 1   & 1   & 1   & 0   & 1     \\
    0   & 1   & 1   & 1   & 1   & 1     \\
    1   & 0   & 0   & 0   & 0   & 0     \\
    1   & 0   & 0   & 0   & 1   & 0     \\
    1   & 0   & 0   & 1   & 0   & 0     \\
    1   & 0   & 0   & 1   & 1   & 1     \\
    1   & 0   & 1   & 0   & 0   & 0     \\
    1   & 0   & 1   & 0   & 1   & 1     \\
    1   & 0   & 1   & 1   & 0   & 1     \\
    1   & 0   & 1   & 1   & 1   & 1     \\
    1   & 1   & 0   & 0   & 0   & 0     \\
    1   & 1   & 0   & 0   & 1   & 1     \\
    1   & 1   & 0   & 1   & 0   & 1     \\
    1   & 1   & 0   & 1   & 1   & 1     \\
    1   & 1   & 1   & 0   & 0   & 1     \\
    1   & 1   & 1   & 0   & 1   & 1     \\
    1   & 1   & 1   & 1   & 0   & 1     \\
    1   & 1   & 1   & 1   & 1   & 1     \\
    \bottomrule
  \end{tabular}
\end{table}
\newpage
\section*{Karnaugh Maps}

\begin{karnaugh-map}[4][4][2][$DE$][$BC$][$A$]
\minterms{7,11,13,14,15,19,21,22,23,25,26,27,28,29,30,31}
\implicant{7}{15}[0,1]
\implicant{15}{11}[0,1]
\implicant{13}{15}[0,1]
\implicant{15}{14}[0,1]
\implicant{5}{15}[1]
\implicant{7}{14}[1]
\implicant{15}{10}[1]
\implicant{13}{11}[1]
\implicant{12}{14}[1]
\implicant{3}{11}[1]
\end{karnaugh-map}

\section*{function}
$$maj(A,B,C,D,E)=\textcolor{red}{CDE}+\textcolor{green}{BDE}+\textcolor{cyan}{BCD}+\textcolor{YellowOrange}{BCE}+\textcolor{blue}{ACE}+\textcolor{magenta}{ACD}
  +\textcolor{cyan}{ABE}+\textcolor{cyan}{ABD}+\textcolor{cyan}{ABC}+\textcolor{cyan}{ADE}$$

\end{document}
