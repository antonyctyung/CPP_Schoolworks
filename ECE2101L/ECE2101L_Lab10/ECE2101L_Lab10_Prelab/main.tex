\documentclass{article}
\usepackage[utf8]{inputenc}
\usepackage{tikz}
\usepackage{siunitx}
\usepackage{graphics}
\usepackage[american,siunitx]{circuitikz}
\usepackage{amsmath}
\usepackage{svg}
\usepackage{booktabs}
\usepackage{float}
\usepackage{xparse, xfp}
\usepackage{graphicx} 
\usepackage{steinmetz}
\usepackage{hyperref}
\usepackage{scalerel}
\usepackage{biblatex}
\addbibresource{Fundamentals_of_Electric_Circuits.bib}
\usetikzlibrary{angles,quotes}
\renewcommand{\thesubsubsection}{\thesubsection.\alph{subsubsection}}
\newcommand{\equal}{=}

\title{ECE 2101L\\Electrical Circuit Analysis II Laboratory\\\,\\Lab 10\\Resonance Circuits\\\,\\Prelab\\}
\author{Choi Tim Antony Yung}
%\author{Choi Tim Antony Yung\\\,\\Willis Nguyen\\Phineas Cozmiuc}

\begin{document}

\clearpage\maketitle
\thispagestyle{empty}
\newpage
\setcounter{page}{1}

\section{Theory of electrical resonance}

According to \textit{Fundamentals of Electric Circuits (2017)}, \textbf{Resonance} is a condition in an RLC circuit in which the capacitive and inductive reactances are equal in magnitude, thereby resulting in a purely resistive impedance. \textbf{Series Resonance} is therefore resonance in a series RLC circuit while \textbf{Parallel Resonance} is resonance in a parallel RLC circuit. The \textbf{Resonance Frequency} for both series and parallel RLC circuit is $\omega_0=\frac{1}{\sqrt{LC}}$\,rad/s. \cite{alexander2017fundamentals}

\section{Series and parallel resonance in RLC circuits}
\begin{center}
    \begin{circuitikz}
        \draw
        (0,0) to[voltage source, v_=$v$] (0,-4)
        (0,0) to[short,i=$i$] ++(1,0) node[ocirc]{}
            to[R=$R$] ++(3,0)
            to[L=$L$] ++(1.25,0) -- ++(1,0)
            to[C=$C$] ++(0,-4) -- (0,-4)
        (1,-4) node[ocirc]{}
        ;
    \end{circuitikz}
\end{center}
The input impedance of the above circuit is 
$$Z=R+\left(\omega L-\frac{1}{\omega C}\right)j$$
The resonance frequency is
$$f_0=\frac{1}{\sqrt{LC}}\text{\,rad/s}=\frac{1}{2\pi\sqrt{LC}}\text{\,Hz}$$
The current under resonance is
$$i(t)=\frac{v(t)}{Z}=\frac{v(t)}{R+\left(\omega_0 L-\frac{1}{\omega_0 C}\right)j}=\frac{v(t)}{R+\left(\sqrt{\frac{L}{C}}-\sqrt{\frac{L}{C}}\right)j}=\frac{v(t)}{R}$$
The magnitude of $i(t)$ under resonance will be smaller comparing to normal condition as the magnitude of Z is at its largest value, therefore $i(t)$ is smallest at resonance frequency comparing to other frequencies.

\newpage

\begin{center}
    \begin{circuitikz}
        \draw
        (0,-3) to[current source, l=$i$] (0,0)
        (0,0) -- ++(1,0) node[ocirc]{} -- ++(1,0) node[circ](R){} -- ++(1.5,0) node[circ](L){} -- ++(1.5,0) to[C=$C$] ++(0,-3) -- (0,-3)
        (1,-0.25) node[below]{$+$}
        (1,-1.5) node{$v$}
        (1,-2.75) node[above]{$-$}
        (1,-3)node[ocirc]{}
        (R) to[R=$R$] ++(0,-3) node[circ]{}
        (L) to[L=$L$] ++(0,-3) node[circ]{}
            
        ;
    \end{circuitikz}
\end{center}
The input impedance of the above circuit is 
$$Z=\frac{1}{\frac{1}{R}+\left(\omega C-\frac{1}{\omega L}\right)j}$$
The resonance frequency is
$$f_0=\frac{1}{\sqrt{LC}}\text{\,rad/s}=\frac{1}{2\pi\sqrt{LC}}\text{\,Hz}$$
The voltage under resonance is
$$v(t)=i(t)Z=\frac{i(t)}{\frac{1}{R}+\left(\omega_0 C-\frac{1}{\omega_0 L}\right)j}=\frac{i(t)}{\frac{1}{R}+\left(\sqrt{\frac{C}{L}}-\sqrt{\frac{C}{L}}\right)j}=i(t)R$$
The magnitude of $v(t)$ under resonance will be larger comparing to normal condition as the magnitude of Z is at its largest value, therefore $v(t)$ is largest at resonance frequency comparing to other frequencies.

\printbibliography

\end{document}